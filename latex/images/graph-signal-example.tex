\begin{tikzpicture}

    % axes
    %\draw[help lines, color=gray!30, dashed] (0,0) grid (12,7);

    % real axis
    \draw[-{Latex[length=3mm, width=3mm]}, very thick] (0,0) -- (0,5) node [right] {$\mathbb{R}$};
    \begin{scope}[every node/.style={anchor=east}]
        \node[color=epfl-perle] at (0,0) {$0$};
        \node[color=epfl-canard] at (0,2) {$1$};
        \node[color=epfl-groseille] at (0,4) {$2$};
    \end{scope}

    % vertices
    \begin{scope}[every node/.style={circle, thick, draw}]
        \node (v1) at (1,0.5) {$v_1$};
        \node (v2) at (2.5,2.5) {$v_2$};
        \node (v3) at (4.5,0.5) {$v_3$};
        \node (v4) at (6,2.5) {$v_4$};
    \end{scope}

    % edges
    \begin{scope}[every edge/.style={thick, draw}]
        \path (v1) edge (v2);
        \path (v2) edge (v3);
        \path (v3) edge (v4);
        \path (v3) edge (v1);
    \end{scope}

    % signal
    \begin{scope}[every node/.style={circle, fill},
                  every edge/.style={dashed, draw}]
        \node[color=epfl-groseille] (sv1) at (1,4.5) {};
        \node[color=epfl-groseille] (sv2) at (2.5,6.5) {};
        \node[color=epfl-groseille] (sv3) at (4.5,4.5) {};
        \node[color=epfl-canard] (sv4) at (6,4.5) {};

        \path[->, color=epfl-groseille] (v1) edge (sv1);
        \path[->, color=epfl-groseille] (v2) edge (sv2);
        \path[->, color=epfl-groseille] (v3) edge (sv3);
        \path[->, color=epfl-canard] (v4) edge (sv4);
    \end{scope}

    % level sets
    \fill[color=epfl-perle, fill=epfl-perle, fill opacity=0.2] (0,0) -- (2.25,3) -- (7,3) -- (4.75,0);
    \fill[color=epfl-canard, fill=epfl-canard, fill opacity=0.2] (0,2) -- (2.25,5) -- (7,5) -- (4.75,2);
    \fill[color=epfl-groseille, fill=epfl-groseille, fill opacity=0.2] (0,4) -- (2.25,7) -- (7,7) -- (4.75,4);

    % signal as a vector
    \node[anchor=center] (x) at (9.75, 4.5) {
        $
        \mathbf{x} = \left [
        \begin{matrix}
            \textcolor{epfl-groseille}{f(v_1)} \\
            \textcolor{epfl-groseille}{f(v_2)} \\
            \textcolor{epfl-groseille}{f(v_3)} \\
            \textcolor{epfl-canard}{f(v_4)}
        \end{matrix}
        \right ]
        = \left [
        \begin{matrix}
            \textcolor{epfl-groseille}{2} \\
            \textcolor{epfl-groseille}{2} \\
            \textcolor{epfl-groseille}{2} \\
            \textcolor{epfl-canard}{1}
        \end{matrix}
        \right ]
        \in \mathbb{R}^{4}
        $
    };

    % graph as a tuple
    \node[anchor=center] (g) at (9.75, 1.5) {
        $
        \begin{matrix}
            \mathcal{G} = (\mathcal{V}, \mathcal{E}) \\
            \mathcal{V} = \left\{ v_1, v_2, v_3, v_4 \right\} \\
            \mathcal{E} = \left\{ (v_1,v_2), (v_2,v_3), (v_3,v_4), (v_3,v_1), \right. \\
            \qquad \left. (v_2,v_1), (v_3,v_2), (v_4,v_3), (v_1,v_3) \right\}
        \end{matrix}
        $
    };

\end{tikzpicture}