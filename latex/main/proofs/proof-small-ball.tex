In this section, $\mathbf{A} \in \mathbb{R}^{m \times n}$ is a random matrix whose rows, $\mathbf{a}_1, \dots, \mathbf{a}_m$, are \acrshort{iid} copies of a random vector $\mathbf{a}$. The reader should recall Definitions \ref{def:min_q_gain}, \ref{def:marginal_tail_function}, and \ref{def:mean_empirical_width} for the minimum $q$-gain, the marginal tail function, and the mean empirical width, respectively. The minimum $q$-gain of $\mathbf{A}$, restricted to a set $\mathcal{S}$, will be seen as a non-negative empirical process induced by the random vectors $\mathbf{a}_1, \dots, \mathbf{a}_m$:
\begin{equation*}
    \gamma_{\min}^{(q)} \left ( \mathcal{S}, \mathbf{A} \right ) = \underset{\mathbf{u} \in \mathcal{S} \cap \operatorname{bd}(\mathbb{B}^{n}_q)}{\inf} \enspace \|\mathbf{A u}\|_q^q = \underset{\mathbf{u} \in \mathcal{S} \cap \operatorname{bd}(\mathbb{B}^{n}_q)}{\inf} \enspace \sum_{i=1}^{m} \left | \langle \mathbf{a}_i, \mathbf{u} \rangle \right |^q.
\end{equation*}

I will then show that for any constants $\xi, t > 0$, and with probability larger than $1 - \exp \left ( \frac{-t^2}{2} \right )$, the lower bound
\begin{equation*}
    \gamma_{\min}^{(q)} \left ( \mathcal{S}, \mathbf{A} \right ) \geq m^{\frac{2 - q}{2q}} \left [ \xi \sqrt{m} Q_{\xi}(\mathbf{a}, \mathcal{S}) - 2 W_{m}(\mathbf{a}, \mathcal{S}) - \xi t \right ]
\end{equation*}
takes place. The argument is taken from Tropp~\cite[Sec. 2.5.5]{tropp2015a}.

\begin{proof}
    \pf\

    \step{}{
        Use, successively, the Lyapunov and Markov inequalities to reach the starting lower bound
        \begin{align*}
            \left ( \frac{1}{m}\sum_{i=1}^{m} \left | \langle \mathbf{a}_i, \mathbf{u} \rangle \right |^q \right )^{1/q} \geq \frac{1}{m} \sum_{i=1}^{m} \left | \langle \mathbf{a}_i, \mathbf{u} \rangle \right | \geq \frac{\xi}{m} \sum_{i=1}^{m} \mathbb{1}_{\{ \left | \langle \mathbf{a}_i, \mathbf{u} \rangle \right | \geq \xi \}}  .
        \end{align*}
    }
    \step{2}{
        Add and subtract $\mathbb{P} \left ( \left \{ \left | \langle \mathbf{a}_i, \mathbf{u} \rangle \right | \geq 2\xi \right \}\right )$ on the \acrshort{rhs} of the inequality in the previous step. Then take the infimum over $\mathcal{S}$ on both sides:
        \begin{align*}
            \underset{\mathbf{u} \in \mathcal{S}}{\inf} \enspace \left ( \frac{1}{m}\sum_{i=1}^{m} \left | \langle \mathbf{a}_i, \mathbf{u} \rangle \right |^q \right )^{1/q} & \geq \xi \overbrace{\underset{\mathbf{u} \in \mathcal{S}}{\inf} \enspace \mathbb{P} \left ( \left \{ \left | \langle \mathbf{a}_i, \mathbf{u} \rangle \right | \geq 2\xi \right \}\right )}^{=: Q_{2\xi}(\mathbf{a}, \mathcal{S})} \\
            & \qquad - \frac{\xi}{m} \underset{\mathbf{u} \in \mathcal{S}}{\sup} \enspace \sum_{i=1}^{m} \left [ \mathbb{P} \left ( \left \{ \left | \langle \mathbf{a}_i, \mathbf{u} \rangle \right | \geq 2\xi \right \}\right ) - \mathbb{1}_{\{ \left | \langle \mathbf{a}_i, \mathbf{u} \rangle \right | \geq \xi \}} \right ]
        \end{align*}
    }
    \step{3}{
       Control the supremum on the new \acrshort{rhs} using the bounded differences inequality \cite[Sec. 6.1]{boucheron2013a}. This is possible because the summands inside the supremum are independent and bounded in magnitude by one. With probability larger than $1 - \exp \left ( -t^2 / 2 \right )$, we then have
        \begin{align*}
            \underset{\mathbf{u} \in \mathcal{S}}{\sup} \enspace \sum_{i=1}^{m} \left [ \mathbb{P} \left ( \left \{ \left | \langle \mathbf{a}_i, \mathbf{u} \rangle \right | \geq 2\xi \right \}\right ) - \mathbb{1}_{\{ \left | \langle \mathbf{a}_i, \mathbf{u} \rangle \right | \geq \xi \}} \right ] & \leq \mathbb{E} \left ( \underset{\mathbf{u} \in \mathcal{S}}{\sup} \enspace \sum_{i=1}^{m} \left [ \mathbb{P} \left ( \left \{ \left | \langle \mathbf{a}_i, \mathbf{u} \rangle \right | \geq 2\xi \right \}\right ) - \mathbb{1}_{\{ \left | \langle \mathbf{a}_i, \mathbf{u} \rangle \right | \geq \xi \}} \right ] \right ) \\
            & \qquad + t \sqrt{m}.
        \end{align*}
    }
    \step{4}{
        It remains to bound the expected supremum to the right. Let $\varepsilon_1, \dots, \varepsilon_m$ be \acrshort{iid} copies of a Rademacher random variable; I claim that the following holds:
        \begin{equation*}
            \mathbb{E} \left ( \underset{\mathbf{u} \in \mathcal{S}}{\sup} \enspace \sum_{i=1}^{m} \left [ \mathbb{P} \left ( \left \{ \left | \langle \mathbf{a}_i, \mathbf{u} \rangle \right | \geq 2\xi \right \}\right ) - \mathbb{1}_{\{ \left | \langle \mathbf{a}_i, \mathbf{u} \rangle \right | \geq \xi \}} \right ] \right ) \leq \frac{2}{\xi} \mathbb{E} \left ( \underset{\mathbf{u} \in \mathcal{S}}{\sup} \enspace \sum_{i=1}^{m} \varepsilon_i \langle \mathbf{a}_i, \mathbf{u} \rangle \right ),
        \end{equation*}
    }
        \begin{proof}
            \pf\
            \step{}{
                Define a ``soft'' indicator function $\psi_{\xi}: \mathbb{R} \to [0,1]$ (see Figure \ref{fig:soft_indicator_function}) as the map
                \begin{equation*}
                    s \mapsto \psi_{\xi}(s) := \left \{
                    \begin{matrix}
                        0, & |s| \leq \xi \\
                        \frac{|s| - \xi}{\xi}, & \xi < |s| \leq 2 \xi \\
                        1, & |s| > 2 \xi \\
                    \end{matrix}
                    \right.
                \end{equation*}
                \begin{figure}[ht]
                    \centering
                    \begin{tikzpicture}

    % axes
    %\draw[help lines, color=gray!30, dashed] (-3,-1) grid (3,4);
    \draw[-{Latex[length=2mm, width=2mm]},thick] (0,0)--(0,3) node[right] {$\psi_{\xi}(s)$};
    \draw[-{Latex[length=2mm, width=2mm]},thick] (-3,0)--(3,0) node[above] {$s$};

    % labels
    \node[anchor=north] at (-2,0) {$-2\xi$};
    \node[anchor=north] at (-1,0) {$-\xi$};
    \node[anchor=north] at (0,0) {$0$};
    \node[anchor=north] at (1,0) {$\xi$};
    \node[anchor=north] at (2,0) {$2\xi$};

    % soft indicator
    \draw[very thick, color=epfl-groseille] (-3,2) -- (-2,2) -- (-1,0) -- (1,0) -- (2,2) -- (3,2);

    % hard indicators
    \draw[thick, dashed, color=epfl-ardoise] (-3,2) -- (-1,2) -- (-1,0) -- (1,0) -- (1,2) -- (3,2);
    \draw[thick, dashed, color=epfl-perle] (-3,2) -- (-2,2) -- (-2,0) -- (2,0) -- (2,2) -- (3,2);

\end{tikzpicture}
                    \caption{``Soft'' indicator function}
                    \label{fig:soft_indicator_function}
                \end{figure}
                We will need two, easily-verifiable properties of this function. First, it is ``sandwiched'' by two indicator functions, $\mathbb{1}_{\{ | \cdot | \geq 2\xi \}} \leq \psi_{\xi}(\cdot) \leq \mathbb{1}_{\{ | \cdot | \geq \xi \}}$. Second, the product map $s \mapsto \xi \psi_{\xi}(s)$ is a contraction.
            }
            \step{}{
                Use the soft indicator function --- and its properties --- to apply a symmetrization procedure \cite[Lemma~6.3]{ledoux2011a}, and then the Rademacher comparison principle \cite[Theorem~4.12]{ledoux2011a} to the expected supremum. As a result, the claim is proved:
                \begin{align*}
                    \mathbb{E} \underset{\mathbf{u} \in \mathcal{S}}{\sup} \enspace \sum_{i=1}^{m} \left [ \mathbb{P} \left ( \left \{ \left | \langle \mathbf{a}_i, \mathbf{u} \rangle \right | \geq \xi \right \}\right ) - \mathbb{1}_{\{ \left | \langle \mathbf{a}_i, \mathbf{u} \rangle \right | \geq \xi \}} \right ] \\
                    & = \mathbb{E} \underset{\mathbf{u} \in \mathcal{S}}{\sup} \enspace \sum_{i=1}^{m} \left [ \mathbb{E} \mathbb{1}_{\{ \left | \langle \mathbf{a}_i, \mathbf{u} \rangle \right | \geq 2\xi \}} - \mathbb{1}_{\{ \left | \langle \mathbf{a}_i, \mathbf{u} \rangle \right | \geq \xi \}} \right ] \\
                    \comment{``sandwiched'' $\psi_{\xi}$} & \leq \mathbb{E} \underset{\mathbf{u} \in \mathcal{S}}{\sup} \enspace \sum_{i=1}^{m} \left [ \mathbb{E} \psi_{\xi}(\langle \mathbf{a}_i, \mathbf{u} \rangle) - \psi_{\xi}(\langle \mathbf{a}_i, \mathbf{u} \rangle) \right ] \\
                    \comment{symmetrization} & \leq 2 \mathbb{E} \underset{\mathbf{u} \in \mathcal{S}}{\sup} \enspace \sum_{i=1}^{m} \varepsilon_i \psi_{\xi}(\langle \mathbf{a}_i, \mathbf{u} \rangle) \\
                    \comment{contraction of $\xi \psi_{\xi}$} & \leq \frac{2}{\xi} \underbrace{\mathbb{E} \underset{\mathbf{u} \in \mathcal{S}}{\sup} \enspace \sum_{i=1}^{m} \varepsilon_i \langle \mathbf{a}_i, \mathbf{u} \rangle}_{=: \sqrt{m} W_{m}(\mathbf{a}, \mathcal{S})} \\
                \end{align*}
            }
            \qedsymbol
        \end{proof}
        \qedstep{The desired lower bound on the minimim $q$-gain functional arises by combining steps \stepref{2}, \stepref{3} and \stepref{4}:
        \begin{equation*}
            \gamma_{\min}^{(q)} \left ( \mathcal{S}, \mathbf{A} \right ) := \underset{\mathbf{u} \in \mathcal{S}}{\inf} \enspace \left ( \frac{1}{m}\sum_{i=1}^{m} \left | \langle \mathbf{a}_i, \mathbf{u} \rangle \right |^q \right )^{1/q} \geq m^{\frac{2 - q}{2q}} \left [ \xi \sqrt{m} Q_{\xi}(\mathbf{a}, \mathcal{S}) - 2 W_{m}(\mathbf{a}, \mathcal{S}) - \xi t \right ].
        \end{equation*}
        }
\end{proof}