\chapter{Dual certificates: \texorpdfstring{\acrshort{kkt}}{KKT} conditions and the golfing scheme}\label{ch:inexact_dual}

As we exit our search for a direct recovery certificate --- cut short by lack of coordinate information on $\mathcal{D}( \|\mathbf{D} \cdot \|_1, \mathbf{x})$ ---, this chapter looks for an alternative in the polar opposite of the descent cone. The subdifferential $\partial \|\mathbf{D} \cdot\|_1(\mathbf{x})$ appears when considering the \acrfull{kkt} conditions for the solutions of
\begin{equation}
    \underset{\mathbf{z} \in \mathbb{R}^{n}}{\min} \| \mathbf{D z} \|_1 \text{ such that } \mathbf{Ax = Az} \tag{P1}.
\end{equation}
One of these conditions relies on the existence of a certain \emph{dual} vector $\mathbf{u} \in \mathbb{R}^{N}$ living in the co-domain of the linear transformation represented by $\mathbf{D}$. The very existence of the dual vector $\mathbf{u}$ can be seen as a recovery certificate for \eqref{eq:l1_interpolation}; the hard task is \emph{proving} that such a vector exists. Nevertheless, I show how to use the \acrshort{kkt} conditions as a blueprint for an iterative scheme producing approximations of the dual vector that are still valid certificates. This idea gives rise to \emph{a version} of the golfing scheme~\cite{gross2011}, popular in \acrlong{cs}, and whose convergence depends on well-behaving tails in random matrices born of the interaction between $\mathbf{D}$ and $\mathbf{A}$. Although powerful, this scheme(by its very construction) applies only to the noiseless, interpolation problem \eqref{eq:l1_interpolation}.~\footnote{This restriction is the main downside of the certificates in this section, as compared to the ones that we could have obtained in the previous chapter.}

In the end, I can reach a sample complexity threshold for the \acrshort{gtv} interpolation under \acrfull{cswr}. More importantly, this threshold explicitly depends on the sampling probabilities $\bm{\pi} = (\pi_1, \dots, \pi_n)$ of \acrshort{cswr}. The corresponding optimal design is then just a corollary of sample complexity result, achieved by minimizing the threshold level \acrlong{wrt} $\bm{\pi}$. Although simple to state, the optimal sampling design is difficult to evaluate in practice, but some approximations of it are examined in the next chapter.


\section{Lagrange dual problem and the \texorpdfstring{\acrshort{kkt}}{KKT} conditions}
The method of Lagrange multipliers~\cite[Ch. 5]{boyd2009} gives us a dual perspective on problem \eqref{eq:l1_interpolation}. First, consider augmenting its objective function in the following way. Let $\bm{\nu} = (\nu_1, \dots, \nu_m) \in \mathbb{R}^{m}$ be a vector with an entry for each of the $m$ implicit equations in $\mathbf{Az} = \mathbf{Ax}$. The numbers $\nu_1, \dots, \nu_m$ will act as the Lagrange multipliers for the equality constraint $\mathbf{Az} - \mathbf{Ax} = \mathbf{0}$. The multipliers augment the objective through the map
\begin{equation}
    \mathbf{z}, \bm{\nu} \mapsto \mathfrak{L}(\mathbf{z}, \bm{\nu}) := \| \mathbf{D z} \|_1 + \left\langle \bm{\nu}, \mathbf{Az} - \mathbf{Ax}\right\rangle,
\end{equation}
The function $\mathfrak{L}: \mathbb{R}^{n} \times \mathbb{R}^{m} \to \mathbb{R}$ is then deemed the Lagrangian of the problem. Second, use the Lagrangian to define the dual problem~\footnote{Taking \eqref{eq:l1_interpolation} as the reference, \emph{primal} problem.}
\begin{equation}
    \underset{\bm{\nu} \in \mathbb{R}^{m}}{\max} \enspace \underset{\mathbf{z} \in \mathbb{R}^{n}}{\min} \enspace \| \mathbf{D z} \|_1 + \left\langle \bm{\nu}, \mathbf{Az} - \mathbf{Ax}\right\rangle \tag{P1-dual},
    \label{eq:l1_interpolation_dual}
\end{equation}
whose objective has an optimal value identical to the one of \eqref{eq:l1_interpolation}~\footnote{This is guaranteed because the primal problem is convex and Slater's condition is satisfied: there is at least one strictly feasible point $\mathbf{z}$, namely $\mathbf{z} \equiv \mathbf{x}$, belonging to the relative interior of the primal's objective~\cite[Sec. 5.2.3]{boyd2009}.}.

The just-defined \eqref{eq:l1_interpolation_dual} is a saddle-point problem, convex in $\mathbf{z}$ and concave~\footnote{It is actually \emph{linear} in $\bm{\nu}$, hence both convex \emph{and} concave.} in $\bm{\nu}$. Variational analysis~\cite[Thm. 8.15]{rockafellar2009} tells us that $(\mathbf{z}^\star, \bm{\nu}^\star)$ is the corresponding saddle-point (or optimal pair) if the inclusions
\begin{align}
    \label{eq:saddle_dual} \mathbf{0} & \in \partial_{\bm{\nu}} \mathfrak{L}(\mathbf{z}^\star, \bm{\nu}^\star), \enspace \text{and}\\
    \label{eq:saddle_primal} \mathbf{0} & \in \partial_{\mathbf{z}} \mathfrak{L}(\mathbf{z}^\star, \bm{\nu}^\star)
\end{align}
take place~\footnote{In Calculus, this is analogous to finding the critical points of a differentiable function in the places where the derivative vanishes.}. The Lagrangian is differentiable with respect to $\bm{\nu}$, so we can unpack \eqref{eq:saddle_dual} as
\begin{align*}
    \mathbf{0} \in \partial_{\bm{\nu}} \mathfrak{L}(\mathbf{z}^\star, \bm{\nu}^\star) & \iff \mathbf{0} = \nabla_{\bm{\nu}} \left\{ \| \mathbf{D}\mathbf{z}^\star \|_1 + \left\langle \cdot, \mathbf{A}\mathbf{z}^\star - \mathbf{Ax}\right\rangle \right\} (\bm{\nu}^\star)\\
    & \iff \mathbf{0} = \mathbf{A}\mathbf{z}^\star - \mathbf{Ax}.
\end{align*}
For the second inclusion, we have to deal with the subdifferential of $\|\mathbf{D} \cdot \|_1$ at $\mathbf{z}^\star$. It helps to recall this set's defining expressions from Proposition \ref{prop:character_subdifferential_l1}, which let us write $\mathbf{z} \in \partial \|\mathbf{D} \cdot \|_1 (\mathbf{z}^\star)$ $\iff$ $\mathbf{z} = \mathbf{D}^{\top} \left[ \operatorname{sign} \left ( \mathbf{Dx} \right ) + \left ( \mathbf{I}_N + \mathbf{P}_{\mathcal{S}} \right ) \mathbf{u} \right]$ for some $\mathbf{u} \in \mathbb{R}^{N}$ satisfying $\left\|(\mathbf{I}_N - \mathbf{P}_{\mathcal{S}}) \mathbf{u}\right\|_\infty \leq 1$. As a result, we can read \eqref{eq:saddle_primal} as
\begin{align*}
    \mathbf{0} \in \partial_{\mathbf{z}} \mathfrak{L}(\mathbf{z}^\star, \bm{\nu}^\star) & \iff \mathbf{0} = \partial \|\mathbf{D} \cdot \|_1 (\mathbf{z}^\star) + \nabla_{\mathbf{z}} \left\{ \left\langle \bm{\nu}^\star, \mathbf{A}\cdot - \mathbf{Ax}\right\rangle \right\} (\mathbf{z}^\star)\\
    & \iff \mathbf{0} = \mathbf{D}^{\top} \left[ \operatorname{sign} \left ( \mathbf{Dx} \right ) + \left ( \mathbf{I}_N + \mathbf{P}_{\mathcal{S}} \right ) \mathbf{u}^\star \right] + \mathbf{A}^{\top} \bm{\nu}^\star,
\end{align*}
where $\mathbf{u}^\star$ is such that $\mathbf{z}^\star = \mathbf{D}^\top \mathbf{u}^\star$ \emph{and} $\left\|(\mathbf{I}_N - \mathbf{P}_{\mathcal{S}}) \mathbf{u}^\star \right\|_\infty \leq 1$.

Together, the unpacked saddle-point expressions form the so-called \acrfull{kkt} conditions for the optimality of the primal-dual pair $(\mathbf{z}^\star, \bm{\nu}^\star)$:
\begin{equation*}
    \mathbf{A}\mathbf{z}^{\star} = \mathbf{Ax}
\end{equation*}
and
\begin{equation*}
    \mathbf{z}^\star = \mathbf{A}^{\top} \bm{\nu}^\star = \mathbf{D}^\top \mathbf{u}^\star \enspace : \enspace \left\{
    \begin{matrix}
        \mathbf{P}_{\mathcal{S}} \mathbf{u}^\star =  \operatorname{sign} \left ( \mathbf{Dx} \right )\\
        \left \| \left ( \mathbf{I}_N - \mathbf{P}_\mathcal{S} \right ) \mathbf{u}^\star \right \|_{\infty} \leq 1
    \end{matrix}
    \right. \enspace .
\end{equation*}
The first of these just restates the interpolation constraint; the second lists the ingredients we will need in the rest of the chapter. Problem \eqref{eq:l1_interpolation} is only succesful if each of its optimal points $\mathbf{z}^\star$ is identical to $\mathbf{x}$. In this case the first of the \acrshort{kkt} conditions is trivially satisfied; let us focus on the second one then.

Note that not much is demanded of the optimal dual point $..bm{\nu}^\star$: it just needs to be different than zero, lest $\mathbf{z}^\star$ be zero as well. The equation $\mathbf{z}^\star = \mathbf{A}^{\top} \bm{\nu}^\star$, in other words, just says that $\mathbf{z}^\star$ should be in the range of $\mathbf{A}^\top$. The only degree of freedom left is vector the $\mathbf{u}^\star$. If there is some such vector simultaneously satisfying
\begin{align}
    \label{eq:kkt0}\mathbf{D}^\top \mathbf{u}^\star \in \operatorname{range} \left( \mathbf{A}^\top \right),\\
    \label{eq:kkt1}\mathbf{P}_{\mathcal{S}} \mathbf{u}^\star =  \operatorname{sign} \left ( \mathbf{Dx} \right )\\
    \label{eq:kkt2}\left \| \left ( \mathbf{I}_N - \mathbf{P}_\mathcal{S} \right ) \mathbf{u}^\star \right \|_{\infty} \leq 1,
\end{align}
and $\mathbf{x} = \mathbf{D}^\top \mathbf{u}^\star$, then it certifies $\mathbf{x}$ as the solution of \eqref{eq:l1_interpolation}. We suddenly see in $\mathbf{u}^\star$ the (dual) certificate we had been looking for.

But before we rush, remember the random nature of the sampling matrix $\mathbf{A}$. It turns impractical the search for a vector $\mathbf{u}^\star$ satisfying an equality constraint like $\mathbf{P}_{\mathcal{S}} \mathbf{u}^\star =  \operatorname{sign} \left ( \mathbf{Dx} \right )$, \emph{while} having a deterministic image $\mathbf{D}^\top \mathbf{u}^\star$ that lies in a random subspace. Fortunately, I show in the next section how some points $\mathbf{u} \in \mathbb{R}^{N}$ can be enough of a recovery certificate despite $\mathbf{P}_{\mathcal{S}} \mathbf{u}$ merely approximating $\operatorname{sign} \left ( \mathbf{Dx} \right )$. This relaxation does not come --- of course --- without a cost, for it demands a more precise control over the interplay between operators $\mathbf{D}$ and $\mathbf{A}$. Still, it pays off, later on, when the defining expressions for these inexact certificates are turned into the blueprint for an effective golfing scheme.


\section{Inexact dual certificates for \texorpdfstring{\acrshort{gtv}}{G-TV} interpolation}\label{sec:inexact_dc}
Inexact dual certificates are a staple of exact recovery studies in \acrlong{cs}, especially when the measurement ensemble is ``structured''~\footnote{As opposed to measurement ensembles like the Gaussian which are considered ``unstructured''}~\cite{adcock2017,boyer2019,candes2011b}. Ever since Cand\`es and Plan~\cite{candes2011b}, the scope of such certificates has been incrementally extended. Using the form of \eqref{eq:l1_interpolation} as a template, we can say that inexact dual certificates were initially shown to exist  only when the sparsifying transform $\mathbf{D}$ was the identity. Then, other proofs started admitting programs with tight frames~\cite{candes2011} or injective operators~\cite{lee2018}. From the perspective of the measurement matrix, the traditionally imposed restraints regarded the covariance structure of $\mathbf{A}$. It was only after Kueng and Gross~\cite{kueng2014} that non-isotropic~\footnote{A random matrix $\mathbf{A} \in \mathbb{R}^{m \times n}$ is isotropic if $\mathbb{E} \left ( \mathbf{A}^\top \mathbf{A}\right ) = \mathbf{I}_{n}$.} measurements could be dealt with.

The lemma that I present next generalizes the progress discussed in the historical account of the previous paragraph. It shows that inexact dual vectors can certify \eqref{eq:l1_interpolation} for a large class of matrices $\mathbf{D}$ and $\mathbf{A}$. In particular, the analysis operator $\mathbf{D}$ need not be injective, as long as its null space intersects trivially with the one of $\mathbf{A}$~\footnote{Recall from Proposition~\ref{prop:trivial_null_necessary} that $\operatorname{null} \left ( \mathbf{D} \right ) \cap \operatorname{null} \left ( \mathbf{A} \right ) = \{ \mathbf{0} \}$ is a \emph{necessary} condition for unique solutions in \eqref{eq:l1_interpolation}.}. Moreover, the random properties of $\mathbf{A}$ are allowed to be regularized by a free parameter in the form of a matrix $\mathbf{B}$.

\begin{lemma}[Inexact Dual Certificate]\label{lem:inexact_dc} Set $\mathbf{M} := \left [ \mathbf{D} (\mathbf{I}_n - \mathbf{B} \mathbf{A}) \mathbf{D}^{+} \right ] ^{\top}$, using some matrix $\mathbf{B} \in \mathbb{R}^{m \times n}$, and let $\mathbf{u}$ be some vector in $\mathbb{R}^{N}$. The point $\mathbf{x} \in \mathbb{R}^{n}$ is certified by $\mathbf{u}$ to be the unique solution of \eqref{eq:l1_interpolation} if all of the following hold:
\begin{align}
\tag{A1} \label{ass:null} \operatorname{null} \left ( \mathbf{D} \right ) \cap \operatorname{null} \left ( \mathbf{A} \right ) = \{ \mathbf{0} \}, \\
\tag{A2} \label{ass:approxPS} \left \| \mathbf{P_\mathcal{S} M P_\mathcal{S}} \right \|_{2} \leq 1 / 3, \\
\tag{A3} \label{ass:boundOffSupport} \underset{k \notin \mathcal{S}}{\max} \left \| \mathbf{P_\mathcal{S} M^{\top} e_k} \right \|_2 \leq 1, \\
\tag{A4} \label{ass:range} \mathbf{D^{\top} u} \in \operatorname{range} \left ( \mathbf{A^{\top}} \right ), \\
\tag{A5} \label{ass:approxSign} \| \mathbf{P}_\mathcal{S} ( \mathbf{u} - \operatorname{sign} \left ( \mathbf{Dx} \right ) ) \|_2 \leq 1 / 3, \\
\tag{A6} \label{ass:approxOffSupport} \| \mathbf{(I_N - P_\mathcal{S}) u} \|_{\infty} \leq 1 / 3.
\end{align}
\end{lemma}

To prove this result, I took inspiration from Boyer \etal~\cite[Appendix A]{boyer2019}. The reader can find the full argument in Appendix~\ref{ap:proof_inexact_dc}, but here is the gist of it.

\begin{proof}
    \pfsketch\ If a perturbation $\mathbf{z} = \mathbf{x + h}$ is a solution of \ref{eq:l1_interpolation}, and the assumptions above hold, I argue then that $\mathbf{h} \equiv \mathbf{0}$ in order to avoid the contradiction $\| \mathbf{D z} \|_1 > \| \mathbf{D x} \|_1$. This line of reasoning follows closely the seminal proof in Cand\`es and Plan~\cite[Lemma 3.2]{candes2011b}, but I make the necessary adaptations due to the non-trivial null space of $\mathbf{D}$.
\end{proof}

As might be expected from the proof sketch, if we set $\mathbf{D} = \mathbf{I}_n$ and $\mathbf{B} = \mathbf{A}^{\top}$ then Lemma~\ref{lem:inexact_dc} reduces to Cand\`es and Plan~\cite[Lemma 3.2]{candes2011b}. The more recent work of Lee \etal has a similar-looking statement~\cite[Lemma 21]{lee2018}, that is nonetheless only valid for $\mathbf{D}$ injective and $\mathbf{A}$ such that $\mathbf{A^\top A}$ is a projection matrix. This can also be seen as a specialization of Lemma~\ref{lem:inexact_dc}, since the injectivity of $\mathbf{D}$ turns Assumption~\ref{ass:null} trivial.

By the way, let me remark some things about the conditions in the lemma. Readers will correctly identify assumptions \ref{ass:range}--\ref{ass:approxOffSupport} as consequences of the \acrshort{kkt} conditions \ref{eq:kkt0}--\ref{eq:kkt2}. The absolute constant of $1/3$ in the assumed inequalities is a presentation choice; examine the proof in Appendix~\ref{ap:proof_inexact_dc} to convince oneself that other numbers could be used --- if they are, at least, strictly less than $1$. The pair \ref{ass:approxPS} and \ref{ass:boundOffSupport}, on the contrary, are artifacts of the proof. They could potentially be replaced if different arguments were devised.

In light of Lemma~\ref{lem:inexact_dc}, I will call any vector $\mathbf{u} \in \mathbb{R}^{N}$ satisfying \ref{ass:range}--\ref{ass:approxOffSupport} an \emph{inexact dual certificate} for $\mathbf{x}$ as \emph{the} solution of \ref{eq:l1_interpolation}. The next section shows how to actually produce such vectors using the lemma's assumptions as guidelines.


\section{The golfing scheme for producing certificates}
\begin{tikzpicture}

    % admissible ball
    \fill[color=epfl-groseille, fill=epfl-groseille, fill opacity=0.2] (1,-2) rectangle (5,2);
    \node[anchor=north west, color=epfl-groseille, opacity=0.5] at (3,-2) {admissible cylinder};

    % axes
    %\draw[help lines, color=gray!30, dashed] (-3,-3) grid (6,5);
    \draw[-{Latex[length=2mm, width=2mm]},very thick] (-1,0)--(6,0) node[right]{$\mathbf{P}_\mathcal{S}(\mathbb{R}^{N})$};
    \draw[-{Latex[length=2mm, width=2mm]},very thick] (0,-3)--(0,3) node[above]{$\left ( \mathbf{I}_N - \mathbf{P}_\mathcal{S} \right )(\mathbb{R}^{N})$};
    \node[anchor=south west] at (5,2) {$\mathbb{R}^{N}$};
    \draw[thick, color=epfl-rouge] (3,-3)--(3,3);

    % positions of points
    \coordinate (0) at (0,0);
    \coordinate (1) at (0.6,-0.6);
    \coordinate (2) at (1.4,1.7);
    \coordinate (dots) at (1.8,1);
    \coordinate (L-2) at (2.2,0.3);
    \coordinate (L-1) at (3.7,1.8);
    \coordinate (L) at (3.3,0.4);
    \coordinate (signDx) at (3,0);
    \coordinate (star) at (3,-1);

    % sign(Dx)
    \draw[fill=black] (signDx) circle (2pt);
    \node[anchor=north west] at (signDx) {$\operatorname{sign} \left ( \mathbf{Dx} \right )$};

    % exact certificate
    \draw[fill=black] (star) circle (2pt);
    \node[anchor=north west] at (star) {$\mathbf{u}^\star$};

    % iterates
    \draw[fill=black] (0) circle (2pt);
    \draw[fill=black] (1) circle (2pt);
    \draw[fill=black] (2) circle (2pt);
    \draw[fill=black] (L-2) circle (2pt);
    \draw[fill=black] (L-1) circle (2pt);
    \draw[fill=black] (L) circle (2pt);

    % arrows linking iterates
    %\draw [->, thick, color=epfl-canard] (0) edge (1) (1) edge (2) (L-2) edge (L-1) (L-1) edge (L);
    \draw [-{Latex[length=2mm, width=2mm]}, thick, color=epfl-canard] (0)--(1);
    \draw [-{Latex[length=2mm, width=2mm]}, thick, color=epfl-canard] (1)--(2);
    \draw [-{Latex[length=2mm, width=2mm]}, thick, color=epfl-canard] (L-2)--(L-1);
    \draw [-{Latex[length=2mm, width=2mm]}, thick, color=epfl-canard] (L-1)--(L);

    % labels of iterates
    \node[anchor=north east] at (0) {$\mathbf{u}^{(0)}$};
    \node[anchor=north] at (1) {$\mathbf{u}^{(1)}$};
    \node at (dots) [rotate=-60.26] {$\dots$};
    \node[anchor=west] at (L-1) {$\mathbf{u}^{(L-1)}$};
    \node[anchor=west] at (L) {$\mathbf{u}^{(L)}$};

\end{tikzpicture}


\section{An optimal vertex-sampling design for \texorpdfstring{\acrshort{gtv}}{G-TV} interpolation}

In the \acrshort{cswr} model, the measurement operator is formed by stacking $m$ standard basis vectors of $\mathbb{R}^{n}$, picked independently at random~\footnote{See Chapter~\ref{ch:graphs_signals_sampling}.}. The picks are determined by \acrshort{iid} copies of a random variable $\omega$ taking values in $[n]$ with probabilities $\mathbb{P} \left ( \left \{  \omega = k \right \}\right ) = \pi_k, \forall k \in [n]$. The \acrshort{iid} copies, $\omega_1, \dots, \omega_m$, form a sampling set $\Omega$ with which we express the sampling matrix as $\mathbf{A} = \left(\mathbf{e}_{\omega_i}^\top\right)_{\omega_i \in \Omega}$.

Skipping some computations~\footnote{It suffices to note that $\mathbf{A}^\top \mathbf{A} = \sum_{i=1}^{m} \mathbf{e}_{\omega_i} \mathbf{e}_{\omega_i}^\top$ and $\mathbb{E} \left ( \mathbf{A}^\top \mathbf{A} \right ) = m \operatorname{diag} \left ( \bm{\pi} \right )$.}, the \acrshort{cswr} model induces the following expression for the matrix $\mathbf{M}$ appearing in Lemma~\ref{lem:tails_golf}:
\begin{equation}
    \mathbf{M} := \left[ \mathbf{D} \left( \mathbf{I}_n - \left[\mathbb{E} \left ( \mathbf{A}^{\top}\mathbf{A} \right ) \right ]^{-1} \mathbf{A}^{\top} \mathbf{A} \right) \mathbf{D}^+ \right]^{\top}
    = \frac{1}{m} \sum_{i=1}^{m} \left [ \mathbf{D} \left ( \mathbf{I}_n - \frac{1}{\pi_{\omega_i}}\mathbf{e}_{\omega_i} \mathbf{e}_{\omega_i}^\top \right ) \mathbf{D}^{+} \right ]^\top.
    \label{eq:M_cswr}
\end{equation}
In words, $\mathbf{M}$ is thus a sum of independent perturbations of the orthogonal projection matrix $\mathbf{D}\mathbf{D}^{+}$ by random, rank-one matrices. Each rank-one matrix is associated with a vertex of the graph via the probabilities $\pi_1, \dots, \pi_n$. By its very construction, a matrix $\mathbf{M}^{(l)}$ only differs then from $\mathbf{M}$ by restricting the limits of the sum in \eqref{eq:M_cswr} to $m_l$ consecutive rows. I will fix each of the chunks $\mathbf{A}^{(1)}, \dots, \mathbf{A}^{(L)}$ of $\mathbf{A}$ in the golfing scheme to be of the same size\footnote{There is no point in doing otherwise for the \acrshort{cswr} model, because the rows of $\mathbf{A}$ are statistically indistinguishable from each other.} (\ie $m_1 = m_2 = \dots = m_L$) in order to write
\begin{equation}
    \mathbf{M}^{\textcolor{epfl-groseille}{(l)}} = \frac{1}{\textcolor{epfl-groseille}{m_1}} \sum_{i = \textcolor{epfl-groseille}{(l-1) \cdot m_{1} + 1}}^{\textcolor{epfl-groseille}{l \cdot m_{1}}} \enspace \left [ \mathbf{D} \left ( \mathbf{I}_n - \frac{1}{\pi_{\omega_i}}\mathbf{e}_{\omega_i} \mathbf{e}_{\omega_i}^\top \right ) \mathbf{D}^{+} \right ]^\top
\end{equation}
at once, for all $l \in [L]$.

The golfing scheme's ability to output an inexact dual certificate depends on the tails of functions of $\mathbf{M}$ and $\{ \mathbf{M}^{(l)}\}_{l \in [L]}$. I will show that these tails are well-behaved if certain moments of the respective matrices are well-behaved. Correspondingly, define the following deterministic parameters (whose notation I borrowed from Boyer~\etal~\cite{boyer2019}).

\begin{definition}
    \begin{align}
        \Theta (\mathcal{S}, \bm{\pi}) & := \underset{i \in [n]}{\max} \enspace \left \| \left [ \mathbf{D} \left ( \mathbf{I}_n - \frac{1}{\pi_{i}}\mathbf{e}_{i} \mathbf{e}_{i}^\top \right ) \mathbf{D}^{+} \right ]^\top \mathbf{P}_{\mathcal{S}}\right \|_{\infty \to \infty} \\
        & = \underset{i \in [n]}{\max} \enspace \underset{k \in [N]}{\max} \enspace \left \| \tilde{\mathbf{e}}_k^{\top} \left [ \mathbf{D} \left ( \mathbf{I}_n - \frac{1}{\pi_{i}}\mathbf{e}_{i} \mathbf{e}_{i}^\top \right ) \mathbf{D}^{+} \right ]^\top \mathbf{P}_{\mathcal{S}}\right \|_{1} \\
        \Upsilon (\mathcal{S}, \bm{\pi}) & := \underset{\|\mathbf{v}\|_\infty \leq 1}{\sup} \enspace \sum_{i=1}^{n} \pi_i \cdot \left \| \left [ \mathbf{D} \left ( \mathbf{I}_n - \frac{1}{\pi_{i}}\mathbf{e}_{i} \mathbf{e}_{i}^\top \right ) \mathbf{D}^{+} \right ]^\top \mathbf{P}_{\mathcal{S}} \mathbf{v} \right \|_{2}^2\\
        \Gamma (\mathcal{S}, \bm{\pi}) & := \max \{ \Theta (\mathcal{S}, \bm{\pi}), \Upsilon (\mathcal{S}, \bm{\pi})\}.
    \end{align}
    \label{def:sample_complexity_parameters}
\end{definition}

The main theorem of this chapter is the next result, which provides a sample complexity threshold for exact recovery in \eqref{eq:l1_interpolation} under \acrshort{cswr} measurements. To prove it, simply call upon certain versions of the Bernstein inequality, as seen in Appendix~\ref{ap:proof_sample_complexity_p1_cswr}. Good Bernstein bounds rely on the good estimation of moments, so I avoided approximations, deferring them to Chapter~\ref{ch:numerical_tour}. The elaborate expressions hidden under $\Gamma (\mathcal{S}, \bm{\pi})$ are a consequence of this choice.

\begin{theorem}[Sample complexity of \eqref{eq:l1_interpolation} under \acrshort{cswr} measurements]\label{thm:sample_complexity_p1_cswr}
    Let $\mathbf{A} \in \mathbb{R}^{m \times n}$ be the measurement matrix in the \acrshort{cswr} model and $\mathbf{D} \in \mathbb{R}^{N \times n}$ be the analysis matrix, denoting $\mathcal{S} := \operatorname{supp}\left ( \mathbf{Dx} \right )$ for some $\mathbf{x} \in \mathbb{R}^{n}$. If $\operatorname{null} \left ( \mathbf{D} \right ) \cap \operatorname{null} \left ( \mathbf{A} \right ) = \{ \mathbf{0} \}$ almost surely, then vector $\mathbf{x}$ is the sole output of \eqref{eq:l1_interpolation}, with probability larger than $1 - \varepsilon$, if
    \begin{equation}
        m \geq 38 \cdot \Gamma(\mathcal{S}, \bm{\pi}) \cdot \log(|\mathcal{S}|) \cdot \log \left ( \frac{63 \cdot N \cdot\log (|\mathcal{S}|)}{\varepsilon} \right ).
    \end{equation}
\end{theorem}

As far as this thesis is concerned, the best sampling design for the \eqref{eq:l1_interpolation} decoder is the one that minimizes its sample complexity. This design --- according to Theorem~\ref{thm:sample_complexity_p1_cswr} --- should therefore minimize $\Gamma(\mathcal{S}, \bm{\pi})$, since this is the only factor in the sample complexity bound that depends on the sampling probabilities $\bm{\pi} = (\pi_1, \dots, \pi_n)$. The next corollary just formalizes this statement.

\begin{corollary}[Optimal \acrshort{cswr} design]\label{cor:opt_samp_design}
    Let $\mathbf{D} \in \mathbb{R}^{N \times n}$ be the analysis matrix for some $\mathbf{x} \in \mathbb{R}^{n}$, yielding the co-support $\mathcal{S} := \operatorname{supp}\left ( \mathbf{Dx} \right )$. The \acrshort{cswr} design that minimizes the number of measurements required by Theorem \ref{thm:sample_complexity_p1_cswr} to exactly recover $\mathbf{x}$ from \eqref{eq:l1_interpolation} with high probability is
    \begin{equation}
        \bm{\pi} = \left \{
        \begin{matrix}
            \text{arg} \enspace \underset{\mathbf{p} \in \mathbb{R}^{n}}{\min} & \Gamma(\mathcal{S}, \mathbf{p}) \\
            \text{subject to} & \mathbf{p} \succeq \mathbf{0} \enspace \text{and} \enspace \langle \mathbf{p}, \mathbf{1} \rangle = 1.
        \end{matrix}
        \right.
    \end{equation}
\end{corollary}

\clearpage

The optimal design --- despite being easy to state --- is not necessarily straightforward to implement. After all, the objective $\Gamma(\mathcal{S}, \bm{\pi})$ is the maximum of two rather complicated expressions, $\Theta(\mathcal{S}, \bm{\pi})$ and $\Upsilon(\mathcal{S}, \bm{\pi})$. Boyer~\etal~\cite{boyer2019}, in a similar situation, suggest looking for common upper bounds to $\Theta(\mathcal{S}, \bm{\pi})$ and $\Upsilon(\mathcal{S}, \bm{\pi})$, and optimize that instead. But finding appropriate upper bounds for our setting goes beyond the scope of this chapter. It is also important to mention the dependence of $\Gamma(\mathcal{S}, \bm{\pi})$ on $\mathcal{S} = \operatorname{supp}\left ( \mathbf{Dx} \right )$~\footnote{Recall that, in the context of piecewise-constant graph signals analysed via the graph gradient, $\mathcal{S}$ is called the jump-set of the signal.}: since $\mathbf{x}$ is the hidden ``ground-truth'' signal, how can one estimate the actions of the projection matrix $\mathbf{P}_{\mathcal{S}}$ without knowing $\mathbf{x}$ a priori? The reader will find some numerical experiments addressing this questions in Chapter \ref{ch:numerical_tour}.


\section{Summary and final notes}

The \acrfull{kkt} conditions of the interpolation problem \eqref{eq:l1_interpolation} reveal that dual certificate vectors arise in the interaction of $\operatorname{range} \left( \mathbf{A} \right)$ and the subdifferential $\partial \|\mathbf{D} \cdot \|_1 (\mathbf{x})$. In fact, merely approximating the \acrshort{kkt} conditions can be enough to guarantee exact recovery. Using Lemma~\ref{lem:inexact_dc} as blueprint, I formulated a golfing scheme that produces potential certificates. Experienced readers might spot how Algorithm~\ref{algo:golf} encompasses other golfing schemes from the literature, derived from particular instances of problem \eqref{eq:l1_interpolation}~\footnote{Problems that have, for example, $\mathbf{D} = \mathbf{I}_n$ and a Gaussian matrix for $\mathbf{A}$.}.

When $\mathbf{A}$ comes from the \acrshort{cswr} model, the success of the golfing scheme demands a number of measurements proportional to a term $\Gamma(\mathcal{S}, \bm{\pi})$. Here lies the explicit connection between the sample complexity of \eqref{eq:l1_interpolation} and the sampling design $\bm{\pi}$. Finding the optimal design is thus a matter of minimizing $\Gamma(\mathcal{S}, \bm{\pi})$, a quantity related to moments of random matrices induced by $\mathbf{D}$ and $\mathbf{A}$. The practical aspects of sampling optimally are discussed in the next chapter.

Lastly, a short note about the absence of the regression version \eqref{eq:l1_regression} in this chapter. Employing the golfing scheme can be suboptimal when dealing with noisy measurements~\cite{krahmer2019}. By this, I mean that the number of measurements predicted by the scheme is knowingly not the best possible for some measurement ensembles. I decided then to restrict this chapter to setting with noiseless samples, hoping that ideas like the ones in Chapter \ref{ch:lower_bound_min_gain} could prove to be effective in the future to study the sample complexity of \acrshort{gtv} regression decoders.

\clearpage

\begin{subappendices}
    \section{Proofs}

    \subsection{Proof of Lemma \ref{lem:inexact_dc}}\label{ap:proof_inexact_dc}
    As a reminder, I follow the strategy of Cand\`es and Plan~\cite[Lemma 3.2]{candes2011b}: assume that some perturbation $\mathbf{x + h}$ is a solution of \eqref{eq:l1_interpolation}, and then show that assumptions \ref{ass:null} -- \ref{ass:approxOffSupport} imply $\mathbf{h = 0}$, lest the contradiction $\|\mathbf{D(x + h)}\|_1 > \|\mathbf{Dx}\|_1$ take place.

\begin{proof}

    \step{}{
        Suppose $\mathbf{x + h}$ is a solution of \eqref{eq:l1_interpolation}. It suffices to consider $\mathbf{h} \in \operatorname{null} \left ( \mathbf{A} \right )$ such that $\mathbf{h} \perp \operatorname{null} \left ( \mathbf{D} \right )$.
    }
        \begin{proof} \pf
            \step{}{
                From the feasilibilty condition, $\mathbf{A(x + h) = Ax \implies h = 0}$. So $\mathbf{h}$ has to be in the null space of $\mathbf{A}$ in order for $\mathbf{x + h}$ to be a solution.
            }
            \step{}{
                If, on top of that, $\mathbf{h} \in \operatorname{null} \left ( \mathbf{D} \right )$, then assumption \ref{ass:null} implies $\mathbf{h = 0}$ and the uniqueness claim holds trivially.
            }
            \step{}{
                Therefore, the interesting perturbations are the ones belonging to the intersection $\operatorname{null} \left ( \mathbf{A} \right ) \cap \operatorname{range} \left( \mathbf{D^{\top}} \right)$.
            }
            \qedsymbol
        \end{proof}

    \step{}{
        Define $\mathbf{g} := \operatorname{sign} \left ( \mathbf{D x} \right ) + (\mathbf{I}_N - \mathbf{P}_\mathcal{S}) \operatorname{sign} \left ( \mathbf{D h} \right ) \in \mathbb{R}^{N}$. Vector $\mathbf{g}$ is a valid subgradient of $\|\cdot\|_1$ at $\mathbf{Dx}$.
    }
        \begin{proof} \pf
            Indeed, we verify
            \begin{align*}
                \mathbf{P}_\mathcal{S} \mathbf{g} & = \operatorname{sign} \left ( \mathbf{Dx} \right ) \\
                \|(\mathbf{I}_N - \mathbf{P}_\mathcal{S}) \mathbf{g}\|_\infty & = \|(\mathbf{I}_N - \mathbf{P}_\mathcal{S}) \operatorname{sign} \left ( \mathbf{D h} \right ) \|_\infty = \left \| \operatorname{sign} ( (\mathbf{I}_N - \mathbf{P}_\mathcal{S}) \mathbf{D h} \right ) \|_\infty \leq 1,
            \end{align*}
        so $\mathbf{g}$ is a valid subgradient, by Proposition \ref{prop:character_subdifferential_l1}.~\qedsymbol
        \end{proof}

    \step{pf:subgrad_lb_1}{
        Use the expression of $\mathbf{g}$ and the fact that it is a subgradient to lower bound $\|\mathbf{D(x + h)}\|_1$ as
        \begin{align*}
            \|\mathbf{D(x + h)}\|_1 & \geq \|\mathbf{D x}\|_1 + \left \langle \mathbf{g, Dh}\right \rangle \\
            & = \|\mathbf{D x}\|_1 + \left \langle \operatorname{sign} \left ( \mathbf{D x} \right ), \mathbf{Dh}\right \rangle + \left \langle (\mathbf{I}_N - \mathbf{P}_\mathcal{S}) \operatorname{sign} \left ( \mathbf{D h} \right ), \mathbf{Dh}\right \rangle \\
            & = \|\mathbf{D x}\|_1 + \left \langle \operatorname{sign} \left ( \mathbf{D x} \right ), \mathbf{Dh}\right \rangle + \left \|(\mathbf{I}_N - \mathbf{P}_\mathcal{S}) \mathbf{Dh} \right \|_1.
        \end{align*}
        We need now to provide estimates to the terms to the right of $\|\mathbf{D x}\|_1$ in the bound above.
    }

    \step{}{
        Let $\mathbf{u} \in \mathbb{R}^{N}$ be any vector satisfying \ref{ass:range}. Then, $\langle \mathbf{u, Dh} \rangle = 0$.
    }
        \begin{proof} \pf\
            Since $\mathbf{h} \in\operatorname{null} \left ( \mathbf{A} \right )$, then $\langle \mathbf{v}, \mathbf{h} \rangle = 0$ for any $\mathbf{v} \in \operatorname{range} \left( \mathbf{A}^{\top} \right)$. But $\mathbf{D}^{\top}\mathbf{u} \in \operatorname{range} \left( \mathbf{A}^{\top} \right)$, so \\ $\langle \mathbf{u, Dh} \rangle = \langle \mathbf{D}^{\top} \mathbf{u}, \mathbf{h} \rangle = 0$.~\qedsymbol
        \end{proof}

    \step{pf:subgrad_lb_2}{
        Add $0 = \langle \mathbf{u, Dh}\rangle - \langle \mathbf{u, Dh}\rangle$ to the bound of step \stepref{pf:subgrad_lb_1}, obtaining
        \begin{align*}
            \|\mathbf{D(x + h)}\|_1 & \geq \|\mathbf{D x}\|_1 + \underbrace{\left \langle \mathbf{u , Dh}\right \rangle}_{=0} - \left \langle \mathbf{u} - \operatorname{sign} \left ( \mathbf{D x} \right ), \mathbf{Dh}\right \rangle + \|(\mathbf{I}_N - \mathbf{P}_\mathcal{S}) \mathbf{Dh} \|_1 \\
            & = \|\mathbf{D x}\|_1 - \left \langle \mathbf{u} - \operatorname{sign} \left ( \mathbf{D x} \right ), \mathbf{Dh}\right \rangle + \|(\mathbf{I}_N - \mathbf{P}_\mathcal{S}) \mathbf{Dh} \|_1.
        \end{align*}
        Next, we upper bound $\left| \left \langle \mathbf{u} - \operatorname{sign} \left ( \mathbf{D x} \right ), \mathbf{Dh}\right \rangle \right|$.
    }

    \step{pf:u_approx_ub}{
        Let $\mathbf{u}$ further abide by assumptions \ref{ass:approxSign} and \ref{ass:boundOffSupport}. Then,
        \begin{equation*}
            \left| \left \langle \mathbf{u} - \operatorname{sign} \left ( \mathbf{D x} \right ), \mathbf{Dh}\right \rangle \right| \leq \frac{1}{3} \|\mathbf{P}_\mathcal{S} \mathbf{Dh}\|_2 + \frac{1}{3} \| (\mathbf{I}_N - \mathbf{P}_\mathcal{S}) \mathbf{Dh}\|_1.
        \end{equation*}
    }
        \begin{proof} \pf
            \step{}{
                Split $\mathbf{u} - \operatorname{sign} \left ( \mathbf{Dx} \right )$ into $\operatorname{range} \left( \mathbf{P}_\mathcal{S} \right)$ and $\operatorname{null} \left ( \mathbf{P}_\mathcal{S} \right )$.
            }
            \step{}{
                Then, using the triangle and H\"older inequalities, along with assumptions \ref{ass:approxSign} and \ref{ass:boundOffSupport}, compute
                \begin{align*}
                    \left|\left \langle \mathbf{u} - \operatorname{sign} \left ( \mathbf{D x} \right ), \mathbf{Dh}\right \rangle\right| & = \left|\left \langle \mathbf{P}_\mathcal{S} (\mathbf{u} - \operatorname{sign} \left ( \mathbf{D x} \right )), \mathbf{Dh}\right \rangle + \left \langle (\mathbf{I}_N - \mathbf{P}_\mathcal{S}) (\mathbf{u} - \operatorname{sign} \left ( \mathbf{D x} \right )), \mathbf{Dh}\right \rangle\right| \\
                    & = \left|\left \langle \mathbf{P}_\mathcal{S} (\mathbf{u} - \operatorname{sign} \left ( \mathbf{D x} \right )), \mathbf{P}_\mathcal{S} \mathbf{Dh}\right \rangle + \left \langle (\mathbf{I}_N - \mathbf{P}_\mathcal{S}) \mathbf{u} , (\mathbf{I}_N - \mathbf{P}_\mathcal{S}) \mathbf{Dh}\right \rangle\right| \\
                    & \leq \left|\left \langle \mathbf{P}_\mathcal{S} (\mathbf{u} - \operatorname{sign} \left ( \mathbf{D x} \right )), \mathbf{P}_\mathcal{S} \mathbf{Dh}\right \rangle \right| + \left | \left \langle \left( \mathbf{I}_N - \mathbf{P}_\mathcal{S} \right) \mathbf{u} , \left( \mathbf{I}_N - \mathbf{P}_\mathcal{S} \right) \mathbf{Dh} \right \rangle \right | \\
                    & \leq \| \mathbf{P}_\mathcal{S} (\mathbf{u} - \operatorname{sign} \left ( \mathbf{D x} \right )) \|_2 \cdot \| \mathbf{P}_\mathcal{S} \mathbf{Dh} \|_2 \\
                    & \qquad + \|(\mathbf{I}_N - \mathbf{P}_\mathcal{S}) \mathbf{u}\|_\infty \cdot \| (\mathbf{I}_N - \mathbf{P}_\mathcal{S}) \mathbf{Dh}\|_1 \\
                    & \leq \frac{1}{3} \|\mathbf{P}_\mathcal{S} \mathbf{Dh}\|_2 + \frac{1}{3} \| (\mathbf{I}_N - \mathbf{P}_\mathcal{S}) \mathbf{Dh}\|_1.
                \end{align*}
            }
            \qedsymbol
        \end{proof}

    \step{pf:subgrad_lb_3}{
        Pair the result from step \stepref{pf:u_approx_ub} with assumption \ref{ass:boundOffSupport}, to read the lower bound from step \stepref{pf:subgrad_lb_2} as
        \begin{equation*}
            \|\mathbf{D(x + h)}\|_1 \geq \|\mathbf{D x}\|_1 - \frac{1}{3} \|\mathbf{P}_\mathcal{S} \mathbf{Dh}\|_2 + \frac{2}{3} \| (\mathbf{I}_N - \mathbf{P}_\mathcal{S}) \mathbf{Dh}\|_1.
        \end{equation*}
        Continue by controlling $\|\mathbf{P}_\mathcal{S} \mathbf{Dh}\|_2$ in terms of $\| (\mathbf{I}_N - \mathbf{P}_\mathcal{S}) \mathbf{Dh} \|_1$.
    }

    \step{pf:psdh_ub}{
        Assumptions \ref{ass:approxPS} and \ref{ass:boundOffSupport} imply
        \begin{equation*}
            \|\mathbf{P}_\mathcal{S} \mathbf{Dh}\|_2 < \frac{3}{2} \| (\mathbf{I}_N - \mathbf{P}_\mathcal{S}) \mathbf{Dh} \|_1.
        \end{equation*}

        \begin{proof} \pf
            \step{pf:some_inverse}{
                Crucially, assumption \ref{ass:approxPS} implies that the matrix $\mathbf{I}_N - \mathbf{P}_\mathcal{S} \mathbf{D}(\mathbf{I}_n - \mathbf{B A}) \mathbf{D}^+ \mathbf{P}_\mathcal{S}$ is \emph{invertible}. Indeed, we can bound the norm of its inverse, using the Neumann series, as
                \begin{align*}
                    \left \| \left [ \mathbf{I}_N - \mathbf{P}_\mathcal{S} \mathbf{D} (\mathbf{I}_n - \mathbf{B A}) \mathbf{D}^+ \mathbf{P}_\mathcal{S} \right ]^{-1} \right \|_2 & = \left \|  \sum_{k=0}^{\infty} \left [ \mathbf{P}_\mathcal{S} \mathbf{D} (\mathbf{I}_n - \mathbf{B A}) \mathbf{D}^+ \mathbf{P}_\mathcal{S} \right ]^k \right \|_2 \\
                    & \leq \sum_{k=0}^{\infty} \left \| \mathbf{P}_\mathcal{S} \mathbf{D} (\mathbf{I}_n - \mathbf{B A}) \mathbf{D}^+ \mathbf{P}_\mathcal{S} \right \|_2^k \\
                    & \leq \sum_{k=0}^{\infty} \frac{1}{3^k} \\
                    & = \frac{3}{2}.
                \end{align*}
            }
            \step{}{
                We can then write the projection matrix $\mathbf{P}_\mathcal{S}$ in the slightly convoluted form $\mathbf{P}_\mathcal{S} = \left [ \mathbf{I}_N - \mathbf{P}_\mathcal{S} \mathbf{D} (\mathbf{I}_n - \mathbf{B A}) \mathbf{D}^+ \mathbf{P}_\mathcal{S} \right ]^{-1} \left [ \mathbf{I}_N - \mathbf{P}_\mathcal{S} \mathbf{D} (\mathbf{I}_n - \mathbf{B A}) \mathbf{D}^+ \right ] \mathbf{P}_\mathcal{S}$.
            }
            \step{}{
                Furthermore, $\mathbf{h} \perp \operatorname{null} \left ( \mathbf{D} \right )$ implies $\mathbf{h} = \mathbf{D}^+\mathbf{D h}$, while $\mathbf{h} \in \operatorname{null} \left ( \mathbf{A} \right )$ implies $\mathbf{BA h} = 0$ for any matrix $\mathbf{B} \in \mathbb{R}^{n \times m}$. Together, these facts entail the identity $\mathbf{h} = (\mathbf{I}_n - \mathbf{B A}) \mathbf{D}^+\mathbf{D h}$.
            }
            \step{}{
                Gathering these observations, and using the shorthand $\mathbf{P}_{\mathcal{S}^{\mathsf{c}}} := \mathbf{I}_N - \mathbf{P}_\mathcal{S}$, we are able to write
                \begin{align*}
                    \|\mathbf{P}_\mathcal{S} \mathbf{Dh}\|_2 & = \left \| \left [ \mathbf{I}_N - \mathbf{P}_\mathcal{S} \mathbf{D} (\mathbf{I}_n - \mathbf{B A}) \mathbf{D}^+ \mathbf{P}_\mathcal{S} \right ]^{-1} \right. \\
                    & \qquad \left. \left [ \mathbf{P}_\mathbb{S} - \mathbf{P}_\mathcal{S} \mathbf{D} (\mathbf{I}_n - \mathbf{B A}) \mathbf{D}^+ \mathbf{P}_\mathcal{S} \right ] \mathbf{Dh}\right \|_2 \\
                    & \leq \left \| \left [ \mathbf{I}_N - \mathbf{P}_\mathcal{S} \mathbf{D} (\mathbf{I}_n - \mathbf{B A}) \mathbf{D}^+ \mathbf{P}_\mathcal{S} \right ]^{-1} \right \|_2 \times \\
                    & \qquad \left \| \mathbf{P}_\mathcal{S} \left [ \mathbf{I}_N - \mathbf{D}(\mathbf{I}_n - \mathbf{B A}) \mathbf{D}^+ \right ] \mathbf{P}_\mathcal{S} \mathbf{D h}\right \|_2 \\
                    & \leq \frac{3}{2} \left \| \mathbf{P}_\mathcal{S} \left [ \mathbf{I}_N - \mathbf{D}(\mathbf{I}_n - \mathbf{B A}) \mathbf{D}^+ \right ] \mathbf{P}_\mathcal{S} \mathbf{D h} \right \|_2 \\
                    & =: \frac{3}{2} \left \| \mathbf{P}_\mathcal{S} \left [ \mathbf{I}_N - \mathbf{D}(\mathbf{I}_n - \mathbf{B A}) \mathbf{D}^+ \right ] (\mathbf{I}_N - \mathbf{P}_{\mathcal{S}^{\mathsf{c}}}) \mathbf{D h}\right \|_2 \\
                    & = \frac{3}{2} \left \| \mathbf{P}_\mathcal{S} \underbrace{\left [ \mathbf{I}_N - \mathbf{D}(\mathbf{I}_n - \mathbf{B A}) \mathbf{D}^+ \right ] \mathbf{D h}}_{=0} \right. \\
                    & \qquad \left. - \mathbf{P}_\mathcal{S} \left [ \mathbf{I}_N - \mathbf{D}(\mathbf{I}_n - \mathbf{B A}) \mathbf{D}^+ \right ] \mathbf{P}_{\mathcal{S}^{\mathsf{c}}} \mathbf{D h} \right \|_2 \\
                    & = \frac{3}{2} \left \| \mathbf{P}_\mathcal{S} \left [ \mathbf{D}(\mathbf{I}_n - \mathbf{B A}) \mathbf{D}^+ \right ] \mathbf{P}_{\mathcal{S}^{\mathsf{c}}} \mathbf{D h} \right \|_2 \\
                    & \leq \frac{3}{2} \sum_{k \in \mathcal{S}^{\mathsf{c}}} \left \| \mathbf{P}_\mathcal{S} \left [ \mathbf{D}(\mathbf{I}_n - \mathbf{B A}) \mathbf{D}^+ \right ] \mathbf{\tilde{e}}_k \right \|_2 \cdot |\mathbf{\tilde{e}}_k ^\top \mathbf{D h} | \\
                    & \leq \frac{3}{2} \underset{k \in \mathcal{S}^{\mathsf{c}}}{\max} \left \| \mathbf{P}_\mathcal{S} \left [ \mathbf{D}(\mathbf{I}_n - \mathbf{B A}) \mathbf{D}^+ \right ] \mathbf{\tilde{e}}_k \right \|_2 \cdot \|\mathbf{P}_{\mathcal{S}^{\mathsf{c}}} \mathbf{D h}\|_1 \\
                    & \leq \frac{3}{2} \|(\mathbf{I}_N - \mathbf{P}_\mathcal{S}) \mathbf{D h}\|_1.
                \end{align*}
            }
            \qedsymbol
        \end{proof}
    }

    \step{pf:subgrad_lb_4}{
        Steps \stepref{pf:subgrad_lb_3} and \stepref{pf:psdh_ub} combine to yield the lower bound
        \begin{equation*}
            \|\mathbf{D(x + h)}\|_1 \geq \|\mathbf{D x}\|_1 + \left ( \frac{2}{3} - \frac{1}{2} \right ) \|(\mathbf{I}_N - \mathbf{P}_\mathcal{S}) \mathbf{D h}\|_1 = \|\mathbf{D x}\|_1 + \frac{1}{6} \|(\mathbf{I}_N - \mathbf{P}_\mathcal{S}) \mathbf{D h}\|_1.
        \end{equation*}
    }

    \step{}{
        But then we must conclude that $(\mathbf{I}_N - \mathbf{P}_\mathcal{S}) \mathbf{D h} = \mathbf{P}_\mathcal{S} \mathbf{D h} = 0$. In other words, $\mathbf{h} \in \operatorname{null} \left ( \mathbf{D} \right )$.
    }
        \begin{proof} \pf
            \step{}{
                Vector $\mathbf{x + h}$ is assumed to be a minimizer of $\|\mathbf{Dz}\|_1$, subject to $\mathbf{Az = Ax}$. Hence, $\|\mathbf{D(x + h)}\|_1 \leq \|\mathbf{Dx}\|_1$, because $\mathbf{x}$ is trivially feasible.
            }
            \step{}{
                In order to avoid contradiction in step \stepref{pf:subgrad_lb_4}, we must then have $\|(\mathbf{I}_N - \mathbf{P}_\mathcal{S}) \mathbf{D h}\|_1 = 0$, meaning $(\mathbf{I}_N - \mathbf{P}_\mathcal{S}) \mathbf{D h} = \mathbf{0}$.
            }
            \step{}{
                The second assertion, $\mathbf{P}_\mathcal{S} \mathbf{D h} = \mathbf{0}$, ultimately follows from the dominance relation $\|\mathbf{P}_\mathcal{S} \mathbf{D h}\|_2 < \frac{3}{2} \|(\mathbf{I}_N - \mathbf{P}_\mathcal{S}) \mathbf{D h}\|_1$.
            }
        \end{proof}

    \qedstep
        \begin{proof}
            The only point $\mathbf{h}$ satisfying both $\mathbf{h} \in \operatorname{null} \left ( \mathbf{D} \right )$ and $\mathbf{h} \perp \operatorname{null} \left ( \mathbf{D} \right )$ is $\mathbf{h} = \mathbf{0}$. Therefore, $\mathbf{x} + \mathbf{h} = \mathbf{x} + \mathbf{0} = \mathbf{x}$ is the only solution of problem \eqref{eq:l1_interpolation}.
        \end{proof}

\end{proof}

\begin{remark}
    The arguably most unnatural step in this proof, $\langle 1 \rangle 8 \langle 2 \rangle 1$, was inspired by a comment in Boyer~\etal~\cite[Appendix A]{boyer2019}. This step is ultimately the reason I could directly adapt the classical proof from Cand\`es~and~Plan~\cite{candes2011b}. To wit, Lee~\etal~\cite[Lemma 21]{lee2018} arrive at a result that is similar to (but slightly weaker than) Lemma~\ref{lem:inexact_dc}, but their proof --- derived from Chen~and~Chi~\cite{chen2014} --- requires splitting their argument into two complementary cases. The first supposes $\|\mathbf{P}_\mathcal{S} \mathbf{D h}\|_2 < \|(\mathbf{I}_N - \mathbf{P}_\mathcal{S}) \mathbf{D h}\|_1$, while the second $\|\mathbf{P}_\mathcal{S} \mathbf{D h}\|_2 > \|(\mathbf{I}_N - \mathbf{P}_\mathcal{S}) \mathbf{D h}\|_1$. Our implicit invertibility assumption in $\left \| \mathbf{P_\mathcal{S} M P_\mathcal{S}} \right \|_{2} \leq 1 / 3$ (a version of which Lee~\etal~\cite{lee2018} \emph{also} require) makes the second case above irrelevant: $\|\mathbf{P}_\mathcal{S} \mathbf{D h}\|_2$ is always dominated by $\|(\mathbf{I}_N - \mathbf{P}_\mathcal{S}) \mathbf{D h}\|_1$.
\end{remark}

    \subsection{Proof of Lemma \ref{lem:tails_golf}}\label{ap:proof_tails_golf}
    All we need to do is verify that the assumptions of Lemma \ref{lem:inexact_dc} hold with high probability. Recall that $\mathbf{w}^{(l)} := \mathbf{P}_\mathcal{S} \left (\operatorname{sign} \left ( \mathbf{Dx} \right ) - \mathbf{u}^{(l)} \right )$ is the error vector at each iteration $l \in [L]$, and the updates of the golfing scheme are given by $\mathbf{u}^{(l)} = \mathbf{u}^{(l-1)} + \left[ \mathbf{I}_N - \mathbf{M}^{(l)} \right] \mathbf{w}^{(l)}$. The rest of the proof is a computation exercise.

\begin{proof}
    \pf\
    \step{}{
        Refer back to expression \ref{eq:golf_error_vec}. If assumptions \eqref{ass:lOnSupport2} hold for each $\mathbf{v} \in \mathbb{R}^{N}$ and $l \in [L]$, then the choice of $L := 1 + \left \lceil \frac{\log |S|}{2 \log 3} \right \rceil$ implies that, with probability at least $1 - \varepsilon/3$,
        \begin{align*}
            \|\mathbf{P}_\mathcal{S} (\mathbf{u}^{(L)} - \operatorname{sign} \left ( \mathbf{Dx} \right ) )\|_2 & \leq \left ( \frac{1}{3} \right )^{L} \sqrt{|S|} \\
            & \leq \frac{1}{3 \sqrt{|\mathcal{S}|} } \sqrt{|\mathcal{S}|} \\
            & \leq \frac{1}{3}.
        \end{align*}
    }
    \step{}{
        Similarly, refferring to inequality \eqref{eq:golf_bound_off_support}, assume \eqref{ass:lOnSupportInf} and \eqref{ass:lOffSupportInf}. For any $L \geq 1$, we have then
        \begin{align*}
            \| (\mathbf{I}_N - \mathbf{P}_\mathcal{S}) \mathbf{u}^{(L)} \|_\infty & \leq \frac{1}{4} \left ( \frac{1 - 1/4^L}{1 - 1/4} \right ) \\
            & = \frac{1}{3} \left ( 1 - \frac{1}{4^L} \right ) \\
            & \leq \frac{1}{3},
        \end{align*}
        with probability at least $1 - \varepsilon/3$.
    }
    \step{}{
        By construction, $\mathbf{u}^{(L)}$ satisfies $\mathbf{D}^{\top}\mathbf{u}^{(L)} \in \operatorname{range} \left ( \mathbf{A^{\top}} \right )$ with probability $1$.
    }
    \step{}{
        Together with assumptions \eqref{ass:approxPSProb} and \eqref{ass:boundOffSupportProb}, we verify all the requirements of Lemma \ref{lem:inexact_dc} with probability at least $1 - \left( \frac{\varepsilon}{3} + \frac{\varepsilon}{3} + \frac{\varepsilon}{3} \right ) = 1 - \varepsilon$.
    }
    \qedstep{With probability larger than $1 - \varepsilon$, vector $\mathbf{u}^{(L)}$ is an inexact dual certificate for $\mathbf{x}$, according to Lemma~\ref{lem:inexact_dc}. Therefore, with the same likelihood, vector $\mathbf{x}$ is the unique solution of problem~\eqref{eq:l1_interpolation}.}
\end{proof}

    \subsection{Proof of Theorem \ref{thm:sample_complexity_p1_cswr}}\label{ap:proof_sample_complexity_p1_cswr}
    \begin{proof}
    \pf\ The argument consists on going through each of the conditions in Lemma \ref{lem:tails_golf}, ensuring that the golfing scheme will produce an inexact dual certificate for the uniqueness of $\mathbf{x}$ as the solution of \eqref{eq:l1_interpolation}. All the operators we have to deal with are sums of bounded, independent random matrices, allowing us to employ the Bernstein inequalities in Appendix \ref{ap:probabilistic_inequalities} to derive the necessary tail bounds.

    \step{bound_spec_norm_PSMPS}{
        I claim that $\mathbb{P} \left ( \left \{  \left \| \mathbf{P}_\mathcal{S} \mathbf{M} \mathbf{P}_\mathcal{S} \right \|_{2 \to 2} > 1 / 3 \right \}\right ) \leq \frac{\varepsilon}{3}$ if $m \geq 24 \cdot \Theta (\mathcal{S}, \bm{\pi}) \cdot \log \left ( \frac{6 |\mathcal{S}|}{\varepsilon} \right )$ .
    }
        \begin{proof}
            \pf\ Write
            \begin{align*}
                \mathbf{X} := \mathbf{P}_\mathcal{S} \mathbf{M} \mathbf{P}_\mathcal{S} = \frac{1}{m} \sum_{i=1}^{m} \underbrace{\mathbf{P}_\mathcal{S}\left [ \mathbf{D} \left ( \mathbf{I}_n - \frac{1}{\pi_{\omega_i}}\mathbf{e}_{\omega_i} \mathbf{e}_{\omega_i}^\top \mathbf{P}_\mathcal{S} \right ) \mathbf{D}^{+} \right ]^\top \mathbf{P}_\mathcal{S}}_{=: \mathbf{X}_i},
            \end{align*}
            a sum of independent, zero-mean random matrices $\{\mathbf{X}_i\}_i$.
            \step{}{
                Bound each $\mathbf{X}_i$ almost surely with $\Theta (\mathcal{S}, \bm{\pi})$:
                \begin{align*}
                    \| \mathbf{X}_i \|_{2 \to 2} & \leq \underbrace{\left \| \mathbf{P}_\mathcal{S} \right \|_{2 \to 2}}_{= 1} \cdot \left \| \left [ \mathbf{D} \left ( \mathbf{I}_n - \frac{1}{\pi_{\omega_i}}\mathbf{e}_{\omega_i} \mathbf{e}_{\omega_i}^\top \right ) \mathbf{D}^{+} \right ]^\top \mathbf{P}_\mathcal{S} \right \|_{2 \to 2}\\
                    & \leq \underset{i \in [n]}{\max} \enspace \left \| \left [ \mathbf{D} \left ( \mathbf{I}_n - \frac{1}{\pi_{i}}\mathbf{e}_{_i} \mathbf{e}_{i}^\top \right ) \mathbf{D}^{+} \right ]^\top \mathbf{P}_\mathcal{S} \right \|_{2 \to 2}\\
                    & \leq \Theta (\mathcal{S}, \bm{\pi}).
                \end{align*}
            }
            \step{}{
                Bound the second moment matrices in the positive definite order:
                \begin{align*}
                    \mathbf{0} \preceq \mathbb{E} \left ( \mathbf{X}_i \mathbf{X}_{i}^\top \right ) & = \mathbf{P}_\mathcal{S} \mathbf{D} \mathbf{D}^{+} \underbrace{\mathbb{E} \left ( \left [ \mathbf{D} \left ( \mathbf{I}_n - \frac{1}{\pi_{\omega_i}}\mathbf{e}_{\omega_i} \mathbf{e}_{\omega_i}^\top \right ) \mathbf{D}^{+} \right ]^\top \mathbf{P}_\mathcal{S} \right ) }_{= \mathbf{0}} \\
                    & \qquad + \mathbb{E} \left ( \frac{1}{\pi_{\omega_i}}\left [ \mathbf{D}^{+} \right ]^{\top} \mathbf{e}_{\omega_i} \mathbf{e}_{\omega_i}^\top \mathbf{D}^{\top}  \mathbf{P}_\mathcal{S} \left [ \mathbf{D} \left ( \mathbf{I}_n - \frac{1}{\pi_{\omega_i}}\mathbf{e}_{\omega_i} \mathbf{e}_{\omega_i}^\top \right ) \mathbf{D}^{+} \right ]^\top \mathbf{P}_\mathcal{S} \right ) \\
                    & \preceq \mathbb{E} \left ( \frac{1}{\pi_{\omega_i}}\left [ \mathbf{D}^{+} \right ]^{\top} \mathbf{e}_{\omega_i} \mathbf{e}_{\omega_i}^\top \mathbf{D}^{\top}  \mathbf{P}_\mathcal{S} \right ) \\
                    & \qquad \times \underset{i \in [n]}{\max} \enspace \left \| \left [ \mathbf{D} \left ( \mathbf{I}_n - \frac{1}{\pi_{i}}\mathbf{e}_{_i} \mathbf{e}_{i}^\top \right ) \mathbf{D}^{+} \right ]^\top \mathbf{P}_\mathcal{S} \right \|_{2 \to 2}\\
                    & \preceq \Theta (\mathcal{S}, \bm{\pi}) \cdot \mathbf{D} \mathbf{D}^{+} \mathbf{P}_\mathcal{S} \\
                    & \preceq \Theta (\mathcal{S}, \bm{\pi}) \cdot \mathbf{I}_N, \enspace \comment{$\mathbf{D} \mathbf{D}^{+}$ is an orthogonal projector}
                \end{align*}
                and, by symmetry, $\mathbb{E} \left ( \mathbf{X}_i^{\top} \mathbf{X}_{i} \right ) \preceq \Theta (\mathcal{S}, \bm{\pi}) \cdot \mathbf{I}_N$.
            }
            \step{}{
                Set the variance parameter $v(\mathbf{X}) := \max \left \{ \mathbb{E} \left ( \mathbf{X} \mathbf{X}^\top \right ), \mathbb{E} \left ( \mathbf{X}^\top \mathbf{X} \right ) \right \} = \frac{1}{m} \Theta (\mathcal{S}, \bm{\pi})$.
            }
            \step{}{
                With these moment bounds, the matrix Bernstein inequality in Lemma \ref{lem:matrix_bern} gives the tail bound
                \begin{align*}
                    \mathbb{P} \left ( \left \{  \left \| \mathbf{P}_\mathcal{S} \mathbf{M} \mathbf{P}_\mathcal{S} \right \|_{2 \to 2} > 1 / 3 \right \}\right ) \leq 2 |\mathcal{S}| \cdot \exp \left ( -\frac{m}{24 \Theta (\mathcal{S}, \bm{\pi})} \right ).
                \end{align*}
            }
            \step{}{
                This probability is less than $\varepsilon / 3$ if $m \geq 24 \cdot \Theta (\mathcal{S}, \bm{\pi}) \cdot \log \left ( \frac{6 |\mathcal{S}|}{\varepsilon} \right )$.
            }
            \qedsymbol
        \end{proof}

    \step{}{
        I claim that $\mathbb{P} \left ( \left \{  \underset{k \notin \mathcal{S}}{\max} \left \| \mathbf{P}_\mathcal{S} \mathbf{M}^{\top} \mathbf{e}_k \right \|_2 > 1 \right \}\right ) \leq\frac{\varepsilon}{3}$ if $m \geq 24 \cdot \Theta (\mathcal{S}, \bm{\pi}) \cdot \log \left ( \frac{6 (N - |\mathcal{S}|)}{\varepsilon} \right )$.
    }
        \begin{proof}
            \pf\ Note the domination relation
            \begin{align*}
                \underset{k \notin \mathcal{S}}{\max} \left \| \mathbf{P}_\mathcal{S} \mathbf{M}^{\top} \mathbf{e}_k \right \|_2 \leq \left \| \mathbf{P}_\mathcal{S} \mathbf{M}^{\top} \left ( \mathbf{I}_N - \mathbf{P}_\mathcal{S} \right ) \right \|_{2 \to 2} = \left \| \left ( \mathbf{I}_N - \mathbf{P}_\mathcal{S} \right ) \mathbf{M} \mathbf{P}_\mathcal{S}  \right \|_{2 \to 2},
            \end{align*}
            implying $\mathbb{P} \left ( \left \{  \underset{k \notin \mathcal{S}}{\max} \left \| \mathbf{P}_\mathcal{S} \mathbf{M}^{\top} \mathbf{e}_k \right \|_2 > 1 \right \}\right ) \leq \mathbb{P} \left ( \left \{  \left \| \left ( \mathbf{I}_N - \mathbf{P}_\mathcal{S} \right ) \mathbf{M} \mathbf{P}_\mathcal{S}  \right \|_{2 \to 2} > 1 \right \}\right )$. The \acrlong{rhs} is less than $\varepsilon / 3$ if $m \geq 24 \cdot \Theta (\mathcal{S}, \bm{\pi}) \cdot \log \left ( \frac{6 (N - |\mathcal{S}|)}{\varepsilon} \right )$, by precisely the same arguments given in step \stepref{bound_spec_norm_PSMPS}.
            \qedsymbol
        \end{proof}

    \step{}{
        For each $l \in [L]$, I claim that $\mathbb{P} \left( \left\{ \left\| \mathbf{P}_\mathcal{S} \mathbf{M}^{(l)} \mathbf{P}_\mathcal{S} \mathbf{v} \right\|_2 > (1/3) \| \mathbf{v} \|_2 \right\} \right) \leq \varepsilon / 3 L$, as long as the row size satisfies $m_l \geq 8 \cdot \max \left \{ 3\Upsilon (\mathcal{S}, \bm{\pi}), \Theta (\mathcal{S}, \bm{\pi})\right \} \cdot \log \left ( \frac{6 L}{\varepsilon} \right )$.
    }
        \begin{proof}
            \pf\ The problem is the same for each $l \in [L]$, so we consider only $l=1$. Fix $\mathbf{v} \in \mathbb{B}_{2}^N$ and write
            \begin{align*}
                \mathbf{P}_\mathcal{S} \mathbf{M}^{(1)} \mathbf{P}_\mathcal{S} \mathbf{v} = \frac{1}{m_1} \sum_{i=1}^{m_1} \underbrace{\mathbf{P}_\mathcal{S}\left [ \mathbf{D} \left ( \mathbf{I}_n - \frac{1}{\pi_{\omega_i}}\mathbf{e}_{\omega_i} \mathbf{e}_{\omega_i}^\top \mathbf{P}_\mathcal{S} \right ) \mathbf{D}^{+} \right ]^\top \mathbf{P}_\mathcal{S} \mathbf{v}}_{=: \mathbf{v}_i},
            \end{align*}
            a sum of independent, zero-mean random vectors $\{\mathbf{v}_i\}_i$.
            \step{}{
                Bound each $\mathbf{v}_i$ almost surely with $\Theta (\mathcal{S}, \bm{\pi})$:
                \begin{align*}
                    \| \mathbf{v}_i \|_{2} & \leq \left \| \left [ \mathbf{D} \left ( \mathbf{I}_n - \frac{1}{\pi_{\omega_i}}\mathbf{e}_{\omega_i} \mathbf{e}_{\omega_i}^\top \right ) \mathbf{D}^{+} \right ]^\top \mathbf{P}_\mathcal{S} \right \|_{2 \to 2} \cdot \underbrace{\left \| \mathbf{v} \right \|_{2}}_{\leq 1} \cdot \\
                    & \leq \Theta (\mathcal{S}, \bm{\pi}).
                \end{align*}
            }
            \step{}{
                Bound the second moment as
                \begin{align*}
                    \mathbb{E} \left ( \| \mathbf{v}_i \|_2^2 \right ) & = \sum_{i=1}^{n} \pi_i \left \| \mathbf{P}_\mathcal{S}\left [ \mathbf{D} \left ( \mathbf{I}_n - \frac{1}{\pi_{\omega_i}}\mathbf{e}_{\omega_i} \mathbf{e}_{\omega_i}^\top \mathbf{P}_\mathcal{S} \right ) \mathbf{D}^{+} \right ]^\top \mathbf{P}_\mathcal{S} \mathbf{v }\right \|_2^2\\
                    & \leq \sum_{i=1}^{n} \pi_i \left \| \left [ \mathbf{D} \left ( \mathbf{I}_n - \frac{1}{\pi_{\omega_i}}\mathbf{e}_{\omega_i} \mathbf{e}_{\omega_i}^\top \mathbf{P}_\mathcal{S} \right ) \mathbf{D}^{+} \right ]^\top \mathbf{P}_\mathcal{S} \mathbf{v }\right \|_2^2\\
                    & \leq \Upsilon (\mathcal{S}, \bm{\pi}).
                \end{align*}
            }
            \step{}{
                Set the variance parameter $\sigma^2 = \frac{1}{m_{1}^2} \sum_{i=1}^{m_1} \mathbb{E} \left ( \| \mathbf{v}_i \|_2^2 \right ) \leq \frac{1}{m_1} \Upsilon (\mathcal{S}, \bm{\pi})$.
            }
            \step{}{
                With these moment bounds, the vector Bernstein inequality in Lemma \ref{lem:vector_bern} gives the tail bound
                \begin{align*}
                    \mathbb{P} \left( \left\{ \left\| \mathbf{P}_\mathcal{S} \mathbf{M}^{(l)} \mathbf{P}_\mathcal{S} \mathbf{v} \right\|_2 > (1/3) \| \mathbf{v} \|_2 \right\} \right) \leq 2 \exp \left ( - \frac{m_1}{8} \min \left \{ \frac{1}{3\Upsilon (\mathcal{S}, \bm{\pi})}, \frac{1}{\Theta (\mathcal{S}, \bm{\pi})}\right \} \right ).
                \end{align*}
            }
            \step{}{
                This probability is less than $\varepsilon / 3L$ if $m_1 \geq 8 \cdot \max \left \{ 3\Upsilon (\mathcal{S}, \bm{\pi}), \Theta (\mathcal{S}, \bm{\pi})\right \} \cdot \log \left ( \frac{6 L}{\varepsilon} \right )$.
            }
            \qedsymbol
        \end{proof}

    \step{}{
        For each $l \in [L]$, I claim that both $\mathbb{P} ( \{ \| \mathbf{P}_\mathcal{S} \mathbf{M}^{(l)} \mathbf{P}_\mathcal{S} \mathbf{v} \|_\infty > (1/4) \| \mathbf{v} \|_\infty \} ) \leq \varepsilon / 3 L$ and its complement $\mathbb{P} ( \{ \| (\mathbf{I}_N - \mathbf{P}_\mathcal{S}) \mathbf{M}^{(l)} \mathbf{P}_\mathcal{S} \mathbf{v} \|_\infty > (1/4) \| \mathbf{v} \|_\infty \} ) \leq \varepsilon / 3 L$ hold, provided that the number of rows satisfies $m_l \geq 8 \cdot \max \left \{ 3\Upsilon (\mathcal{S}, \bm{\pi}), \Theta (\mathcal{S}, \bm{\pi})\right \} \cdot \log \left ( \frac{6 N}{\varepsilon} \right )$.
    }
        \begin{proof}
            \pf\ Once again, the problem is the same for each $l \in [L]$, so we consider only $l=1$. Fix $k \in [N]$, some $\mathbf{v} \in \mathbb{B}_{\infty}^N$, and write
            \begin{align*}
                X := \left \langle \tilde{\mathbf{e}}_k, \mathbf{M}^{(l)} \mathbf{P}_\mathcal{S} \mathbf{v} \right \rangle = \frac{1}{m_1} \sum_{i=1}^{m_1}  \underbrace{\left \langle \tilde{\mathbf{e}}_k, \left [ \mathbf{D} \left ( \mathbf{I}_n - \frac{1}{\pi_{\omega_i}}\mathbf{e}_{\omega_i} \mathbf{e}_{\omega_i}^\top \mathbf{P}_\mathcal{S} \right ) \mathbf{D}^{+} \right ]^\top \mathbf{P}_\mathcal{S} \mathbf{v} \right \rangle}_{=: X_i}
            \end{align*}
            a sum of independent, zero-mean random variables $\{X_i\}_i$.
            \step{}{
                Bound each $X_i$ almost surely with $\Theta (\mathcal{S}, \bm{\pi})$:
                \begin{align*}
                    |X_i| & = \left \| \tilde{\mathbf{e}}_k^\top \left [ \mathbf{D} \left ( \mathbf{I}_n - \frac{1}{\pi_{\omega_i}}\mathbf{e}_{\omega_i} \mathbf{e}_{\omega_i}^\top \mathbf{P}_\mathcal{S} \right ) \mathbf{D}^{+} \right ]^\top \mathbf{P}_\mathcal{S} \right \|_1 \cdot \underbrace{\| \mathbf{v} \|_{\infty}}_{\leq 1}\\
                    & \leq \Theta (\mathcal{S}, \bm{\pi}).
                \end{align*}
            }
            \step{}{
                Bound the second moment as
                \begin{align*}
                    \mathbb{E} \left ( | X_i |^2 \right ) & = \sum_{i=1}^{n} \pi_i \left | \left \langle \tilde{\mathbf{e}}_k, \left [ \mathbf{D} \left ( \mathbf{I}_n - \frac{1}{\pi_{\omega_i}}\mathbf{e}_{\omega_i} \mathbf{e}_{\omega_i}^\top \mathbf{P}_\mathcal{S} \right ) \mathbf{D}^{+} \right ]^\top \mathbf{P}_\mathcal{S} \mathbf{v} \right \rangle \right |^2\\
                    & \leq \Upsilon (\mathcal{S}, \bm{\pi}).
                \end{align*}
            }
            \step{}{
                Set the variance parameter $\sigma^2 = \frac{1}{m_1^2} \sum_{i=1}^{m_1} \mathbb{E} \left ( | X_i |^2 \right ) \leq \frac{1}{m_1} \Upsilon (\mathcal{S}, \bm{\pi})$.
            }
            \step{}{
                The scalar Bernstein inequality in Lemma \ref{lem:scalar_bern} gives the tail bound
                \begin{align*}
                    \mathbb{P} \left( \left\{ \left \langle \tilde{\mathbf{e}}_k, \mathbf{M}^{(l)} \mathbf{P}_\mathcal{S} \mathbf{v} \right \rangle > (1/4) \| \mathbf{v} \|_2 \right\} \right) \leq 2 \exp \left ( - \frac{3m_1}{32} \min \left \{ \frac{1}{4\Upsilon (\mathcal{S}, \bm{\pi})}, \frac{1}{\Theta (\mathcal{S}, \bm{\pi})}\right \} \right )
                \end{align*}
                for each fixed $k \in [N]$.
            }
            \step{}{
                Taking the union bound over $\mathcal{S}$ and then over $N \setminus [N]$ yields in turn
                \begin{align*}
                    \mathbb{P} \left( \left\{ \left \langle \tilde{\mathbf{e}}_k, \mathbf{M}^{(l)} \mathbf{P}_\mathcal{S} \mathbf{v} \right \rangle > (1/4) \| \mathbf{v} \|_2 \right\} \right) & \leq 2 |\mathcal{S}| \times \\
                    & \qquad \exp \left ( - \frac{3m_1}{32} \min \left \{ \frac{1}{4\Upsilon (\mathcal{S}, \bm{\pi})}, \frac{1}{\Theta (\mathcal{S}, \bm{\pi})}\right \} \right )\\
                    \mathbb{P} \left( \left\{ \left \langle \tilde{\mathbf{e}}_k, \mathbf{M}^{(l)} \mathbf{P}_\mathcal{S} \mathbf{v} \right \rangle > (1/4) \| \mathbf{v} \|_2 \right\} \right) & \leq 2 (N - |\mathcal{S}|) \times \\
                    & \qquad \exp \left ( - \frac{3m_1}{32} \min \left \{ \frac{1}{4\Upsilon (\mathcal{S}, \bm{\pi})}, \frac{1}{\Theta (\mathcal{S}, \bm{\pi})}\right \} \right )
                \end{align*}
            }
            \step{}{
                Both these probabilities are less than $\varepsilon / 3 L$ if $m_1 \geq \frac{32}{3} \cdot \max \left \{ 4\Upsilon (\mathcal{S}, \bm{\pi}), \Theta (\mathcal{S}, \bm{\pi})\right \} \cdot \log \left ( \frac{6 N L}{\varepsilon} \right )$.
            }
        \end{proof}

    \step{}{
        We now call upon the definition of $\Gamma (\mathcal{S}, \bm{\pi}) := \max \left \{ 4\Upsilon (\mathcal{S}, \bm{\pi}), \Theta (\mathcal{S}, \bm{\pi})\right \}$. All requirements in Lemma \ref{lem:tails_golf} depending on matrices $\mathbf{M}^{(l)}$ are simultaneously attained if
        \begin{align*}
            m = \sum_{l=1}^{L} m_l \geq \frac{32L}{3} \cdot \Gamma (\mathcal{S}, \bm{\pi}) \cdot \log \left ( \frac{6 N L}{\varepsilon} \right ),
        \end{align*}
        whereas the requirements depending on matrix $\mathbf{M}$ are enforced if
        \begin{align*}
            m \geq 24 \cdot \Theta (\mathcal{S}, \bm{\pi}) \cdot \log \left ( \frac{6 N}{\varepsilon} \right ).
        \end{align*}
        Recalling that $L \geq 2 + \left \lceil \frac{\log |S|}{2 \log 3} \right \rceil$, if suffices then to set
        \begin{align*}
            m \geq \frac{32}{3} \left ( 2 + \frac{\log |\mathcal{S}|}{2 \log(3)} \right ) \cdot \Gamma(\mathcal{S}, \bm{\pi}) \cdot \log(|\mathcal{S}|) \cdot \log \left ( \frac{6 \cdot N \cdot \left ( 2 + \frac{\log |\mathcal{S}|}{2 \log(3)} \right )}{\varepsilon} \right ),
        \end{align*}
        which can be simplified to $m \geq 38 \cdot \Gamma(\mathcal{S}, \bm{\pi}) \cdot \log(|\mathcal{S}|) \cdot \log \left ( \frac{63 \cdot N \cdot\log (|\mathcal{S}|)}{\varepsilon} \right )$ if we assume $|\mathcal{S}| \geq 3$.~\footnote{We do not lose in doing so, since a co-support $\mathcal{S} = \operatorname{supp}\left ( \mathbf{Dx} \right )$ of $3$ is orders of magnitude below what is normally encountered in applications.}
    }

    \qedstep{All the conditions of Lemma \ref{lem:tails_golf} hold simutaneously with probability larger than $1 - \varepsilon$. Therefore --- with the same likelihood --- the golfing scheme certifies $\mathbf{x}$ to be the unique solution of \eqref{eq:l1_interpolation}.}
\end{proof}
\end{subappendices}