Inexact dual certificates are a staple of exact recovery studies in \acrlong{cs}, especially when the measurement ensemble is ``structured''~\footnote{As opposed to measurement ensembles like the Gaussian which are considered ``unstructured''}~\cite{adcock2017,boyer2019,candes2011b}. Ever since Cand\`es and Plan~\cite{candes2011b}, the scope of such certificates has been incrementally extended. Using the form of \eqref{eq:l1_interpolation} as a template, we can say that inexact dual certificates were initially shown to exist only when the sparsifying transform $\mathbf{D}$ was the identity. Then, other proofs started admitting programs with tight frames~\cite{candes2011} or injective operators~\cite{lee2018}. From the perspective of the measurement matrix, the traditionally imposed restraints regarded the covariance structure of $\mathbf{A}$. It was only after Kueng and Gross~\cite{kueng2014} that non-isotropic~\footnote{A random matrix $\mathbf{A} \in \mathbb{R}^{m \times n}$ is isotropic if $\mathbb{E} \left ( \mathbf{A}^\top \mathbf{A}\right ) = \mathbf{I}_{n}$.} measurements could be dealt with.

The lemma that I present next generalizes the progress discussed in the historical account of the previous paragraph. It shows that inexact dual vectors can certify \eqref{eq:l1_interpolation} for a large class of matrices $\mathbf{D}$ and $\mathbf{A}$. In particular, the analysis operator $\mathbf{D}$ need not be injective, as long as its null space intersects trivially with the one of $\mathbf{A}$~\footnote{Recall from Proposition~\ref{prop:trivial_null_necessary} that $\operatorname{null} \left ( \mathbf{D} \right ) \cap \operatorname{null} \left ( \mathbf{A} \right ) = \{ \mathbf{0} \}$ is a \emph{necessary} condition for unique solutions in \eqref{eq:l1_interpolation}.}. Moreover, the random properties of $\mathbf{A}$ are allowed to be regularized by a free parameter in the form of a matrix $\mathbf{B}$.

\begin{lemma}[Inexact Dual Certificate]\label{lem:inexact_dc} Set $\mathbf{M} := \left [ \mathbf{D} (\mathbf{I}_n - \mathbf{B} \mathbf{A}) \mathbf{D}^{+} \right ] ^{\top}$, using some matrix $\mathbf{B} \in \mathbb{R}^{m \times n}$, and let $\mathbf{u}$ be some vector in $\mathbb{R}^{N}$. The point $\mathbf{x} \in \mathbb{R}^{n}$ is certified by $\mathbf{u}$ to be the unique solution of \eqref{eq:l1_interpolation} if all of the following hold:
\begin{align}
\tag{A1} \label{ass:null} \operatorname{null} \left ( \mathbf{D} \right ) \cap \operatorname{null} \left ( \mathbf{A} \right ) = \{ \mathbf{0} \}, \\
\tag{A2} \label{ass:approxPS} \left \| \mathbf{P_\mathcal{S} M P_\mathcal{S}} \right \|_{2} \leq 1 / 3, \\
\tag{A3} \label{ass:boundOffSupport} \underset{k \notin \mathcal{S}}{\max} \left \| \mathbf{P_\mathcal{S} M^{\top} e_k} \right \|_2 \leq 1, \\
\tag{A4} \label{ass:range} \mathbf{D^{\top} u} \in \operatorname{range} \left ( \mathbf{A^{\top}} \right ), \\
\tag{A5} \label{ass:approxSign} \| \mathbf{P}_\mathcal{S} ( \mathbf{u} - \operatorname{sign} \left ( \mathbf{Dx} \right ) ) \|_2 \leq 1 / 3, \\
\tag{A6} \label{ass:approxOffSupport} \| \mathbf{(I_N - P_\mathcal{S}) u} \|_{\infty} \leq 1 / 3.
\end{align}
\end{lemma}

To prove this result, I took inspiration from Boyer \etal~\cite[Appendix A]{boyer2019}. The reader can find the full argument in Appendix~\ref{ap:proof_inexact_dc}, but here is the gist of it.

\begin{proof}
    \pfsketch\ If a perturbation $\mathbf{z} = \mathbf{x + h}$ is a solution of \ref{eq:l1_interpolation}, and the assumptions above hold, I argue then that $\mathbf{h} \equiv \mathbf{0}$ in order to avoid the contradiction $\| \mathbf{D z} \|_1 > \| \mathbf{D x} \|_1$. This line of reasoning follows closely the seminal proof in Cand\`es and Plan~\cite[Lemma 3.2]{candes2011b}, but I make the necessary adaptations due to the non-trivial null space of $\mathbf{D}$.
\end{proof}

As might be expected from the proof sketch, if we set $\mathbf{D} = \mathbf{I}_n$ and $\mathbf{B} = \mathbf{A}^{\top}$ then Lemma~\ref{lem:inexact_dc} reduces to Cand\`es and Plan~\cite[Lemma 3.2]{candes2011b}. The more recent work of Lee \etal has a similar-looking statement~\cite[Lemma 21]{lee2018}, that is nonetheless only valid for $\mathbf{D}$ injective and $\mathbf{A}$ such that $\mathbf{A^\top A}$ is a projection matrix. This can also be seen as a specialization of Lemma~\ref{lem:inexact_dc}, since the injectivity of $\mathbf{D}$ turns Assumption~\ref{ass:null} trivial.

By the way, let me remark on some things about the conditions in the lemma. Readers will correctly identify assumptions \ref{ass:range}--\ref{ass:approxOffSupport} as consequences of the \acrshort{kkt} conditions \ref{eq:kkt0}--\ref{eq:kkt2}. The absolute constant of $1/3$ in the assumed inequalities is a presentation choice; examine the proof in Appendix~\ref{ap:proof_inexact_dc} to convince oneself that other numbers in $(0,1)$ could be used. The pair \ref{ass:approxPS} and \ref{ass:boundOffSupport}, on the contrary, are artifacts of the proof. They could potentially be replaced if different arguments were devised.

In light of Lemma~\ref{lem:inexact_dc}, I will call any vector $\mathbf{u} \in \mathbb{R}^{N}$ satisfying \ref{ass:range}--\ref{ass:approxOffSupport} an \emph{inexact dual certificate} for $\mathbf{x}$ as \emph{the} solution of \ref{eq:l1_interpolation}. The next section shows how to actually produce such vectors using the lemma's assumptions as guidelines.